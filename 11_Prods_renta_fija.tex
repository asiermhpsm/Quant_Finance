\section{Productos de renta fija}



\subsection{Contratos simples de renta fija}

\subsubsection{Bonos de cupón cero}
Es un contrato que paga una contidad fija conocida (\textbf{principal}) en una fecha determinada (\textbf{maturity date}) $T$. Dicho valor se debe actualizar para calcular su precio.



\subsubsection{Bonos con cupón (coupon-bearing bond)}
Además de pagar el principal a fecha de vencimiento, paga \textbf{cupones} en ciertas fechas preestablecidas. Los cupones suelen ser un porcentaje del principal y se suelen pagar en periodos regulares.



\subsubsection{Money Market Account}
Cuentas de dinero (p.e.\ en el banco) que acumulan intereses compuestos de vez en cuando. El interés suele ser a corto plazo e impredecible, por lo que es ``arriesgado''. Tiene la ventaja de ser flexible (lo puedes mover cuando quieras).



\subsubsection{Floating Rate Bonds}
Los bonos de tasa flotante son bonos cuya tasa de interés está ligada a un índice de referencia  como el LIBOR (p.e.\ se puede recibir LIBOR+1\%). Protegen al inversor contra subidas de interés pero tiene poca flexibilidad (hay que esperar a vencimiento) y hay incertidumbre de lo que se va a recibir.



\subsubsection{Forward Rate Agreements (FRA)}
Se fija una tasa de interés fija sobre un principal. Una parte paga el principal en $T_1$ y la otra parte lo devuelve con los intereses acordados en $T_2 > T_1$. El valor del contrato al inicio suele no ser cero, por lo que puede haber un pago inicial entre las partes.



\subsubsection{Repos}
Es un acuerdo de recompra. Consiste en vender un activo financiero a otra parte y acordar recomprarlo en una fecha y cantidad fijada. El precio de recompra suele ser mayor que el de venta, y la diferencia implica un tipo de interés llamado \textbf{repo rate}. El más común es el \textit{overnight repo}, que se renegocia diariamente. Si el acuerdo dura más de 30 días se denomina \textit{term repo}. Un \textbf{reverse repo} es la operación inversa: la compra de un valor con el compromiso de venderlo posteriormente.



\subsubsection{Bonos separables (STRIPS)}
‘Separate Trading of Registered Interest and Principal of Securities’. Consisten en separar los cupones y el principal de los bonos tradicionales, creando así bonos artificiales de cupón cero con vencimientos más largos de los que normalmente estarían disponibles.

Por ejemplo se ha comprado un bono con cupones que dan 5\euro\ al año. Se pueden vender cada uno de esos cupones como bonos de cupón cero, y valdrían los 5\euro\ actualizados.



\subsubsection{Amortización}
El principal va disminuyendo poco a poco durante la vida del contrato y los intereses se calculan sobre el principal pendiente. 

La amortización puede ser fija (con un calendario conocido de antemano) o depender de algún índice (por ejemplo, si el índice sube, el principal se amortiza más rápido).

\subsubsection{Cláusula de rescate anticipado (Call Provision)}
Es una cláusula que se pone a contratos de renta fija que permite al emisor recomprar el contrato en ciertas fechas o periodos por un importe preestablecido. Esto reduce el valor del contrato para el inversor.






\subsection{Mercado internacional de bonos}
\begin{itemize}
    \item \textbf{USA}: 
    \begin{itemize}
        \item \textbf{Bill}: maturity menor que un año y normalmente sin cupón.
        \item \textbf{Note}: maturity entre 2 y 10 años, con cupón cada 6 meses.
        \item \textbf{Bond}: maturity mayor que 10 años, con cupón cada 6 meses.
        \item \textbf{Yankees}: comerciados en USA por instituciones extranjeras.
    \end{itemize}
    \item \textbf{UK}: los emitidos por el gobierno se llaman \textbf{gilts}. Incluyen bonos \textit{callable, irredeemable, convertible} o \textit{index-linked} ligados a Retail Price Index (RPI). Más adelante se explicará que es cada cosa.
    \item \textbf{Japón}: \textbf{Japanese Government Bonds (JGBs)} pueden ser a corto plazo (letras del tesoro, sin cupones), plazo medio (con o sin cupones), largo plazo (maturity de 10 años, cupones cada 6 meses) o plazo muy largo (maturity de 20 años, cupones cada 6 meses). Los emitidos en yenes por instituciones extranjeras son bonos \textbf{Samurai}.
\end{itemize}




\subsection{Conteo de días}
Algunas maneras de contar los días entre dos fechas:
\begin{itemize}
    \item \textbf{Actual/Actual}: número de días normal que hay en el calendario.
    \item \textbf{30/360}: cada mes tiene 30 días y el año tiene 360 días
    \item \textbf{Actual/360}: cada mes tiene los días que toca, pero el año tiene 360 días.
\end{itemize}













