\section{Bonos convertibles}
El \textbf{bono convertible (CB)} sobre una acción paga cupones específicos con devolución del capital al vencimiento, a menos que previamente el titular haya convertido el bono en el activo subyacente.
Por lo tanto, un bono convertible tiene las características de un bono ordinario, pero con la ventaja adicional de que puede, en el momento que elija el titular, canjearse por un activo específico. Este canje se denomina conversión. 

Por lo tanto, un bono convertible esta limitado por:
\begin{itemize}
    \item El valor de la conversión: que es el importe recibido si el bono se convierte inmediatamente (independientemente de si esto es óptimo). Esto es:
    \begin{equation*}
        \text{conversion value} = \text{market price of stock} \times \text{conversion ratio}
    \end{equation*}
    \item Su valor como bono corporativo, con capital final y cupones durante su vigencia. Esto se denomina \textbf{straight value} (valor directo).
\end{itemize}

Este último punto muestra cómo existen problemas de riesgo crediticio en la fijación de precios de los CBs. Estos problemas de riesgo crediticio son muy importantes, pero no se analizará en detalle aquí. Se supondrá que no existe riesgo de impago. El riesgo crediticio se analiza en profundidad más adelante.









\subsubsection{Uso de los bonos convertibles en el mundo real}
Las empresas emiten bonos para captar capital. Los bonos corporativos se presentan en una amplia variedad. Por un lado, están los bonos vainilla, con cupón fijo y reembolso del principal al vencimiento. La mayoría de los bonos corporativos tienen una cláusula de rescate que permite a la empresa emisora rescatarlos antes del vencimiento. Esto ocurriría si las tasas de interés cayeran lo suficiente por debajo de las tasas de mercado vigentes en ese momento. Algunos bonos tienen cláusulas de venta que permiten al tenedor revenderlos al emisor antes del vencimiento. Debido a la posibilidad de impago, los bonos corporativos se clasifican según su prioridad para recibir activos en caso de quiebra. Los bonos convertibles son simplemente otro tipo de bono corporativo. Desde la perspectiva del emisor, presentan ventajas sobre los bonos vainilla, ya que, si la empresa tiene éxito, el precio de sus acciones subirá, el tenedor convertirá el bono y el emisor emitirá nuevas acciones de la empresa. Emitir nuevas acciones puede considerarse una mejor opción que el pago continuo de intereses. El tenedor del bono compra el bono convertible para obtener exposición a la empresa y cierta protección contra pérdidas. Por supuesto, estamos en el mundo del papel comercial y siempre existirá el riesgo de impago.

La mayoría de los bonos convertibles son emitidos por empresas pequeñas. A menudo, el cupón es bajo. La pequeña empresa tiene obligaciones con tasas de interés bajas a corto plazo. A largo plazo, si tiene éxito, cederá acciones a inversores satisfechos. Por otro lado, los bonos convertibles pueden ser emitidos por empresas más especulativas que se verían tentadas a asumir riesgos con el capital así obtenido.

\underline{¿Cuál es su desventaja?}
El bono convertible tiene características que hacen que a veces se comporte como una acción y a veces como un bono. La característica de conversión de los bonos convertibles también hace que estos contratos sean similares a las opciones americanas. La cuestión de cuándo ejercer una opción americana es muy similar a la de cuándo convertir un bono convertible. Es esta ``opcionalidad'' la que añade valor al bono convertible. Sin embargo, los bonos convertibles se diferencian de las opciones en que, al convertirse, se emiten nuevas acciones. El CB es un ejemplo de instrumento híbrido ya que tiene características tanto de capital como de deuda.

El principal problema con estos bonos es que generalmente los emiten compañias con un credit risk malo. Al vender bonos que luego pueden convertirse en acciones, el emisor puede obtener un cupón más bajo del que podría esperarse en otras circunstancias.

\underline{¿Por qué emitir un bono convertible?}
Las empresas que necesitan capital tienen muchas opciones disponibles. Estas opciones tienen dos temas comunes: Emitir capital; Emitir deuda.
\begin{itemize}
    \item Emitir capital: Diluye las ganancias por acción, pero tiene bajos costos iniciales de financiamiento.
    \item Emitir deuda: No diluye las ganancias por acción, pero puede tener altos costos iniciales de financiamiento.
\end{itemize}
La otra posibilidad es emitir un instrumento híbrido con ambas características: el bono convertible.
\begin{itemize}
    \item El bono puede venderse con un cupón más bajo que un bono simple con el mismo vencimiento y precio. (O, equivalentemente, puede venderse a un precio más alto con el mismo cupón).
    \item No diluye las ganancias por acción hasta que el bono se convierte y se emiten nuevas acciones.
    \item Si el bono se convierte, no es necesario reembolsar el capital.
\end{itemize}
El emisor típico de bonos convertibles tiene altas necesidades de efectivo, quizás con un uso muy rápido de efectivo y baja calidad crediticia. Pueden ser startups. En EE.\ UU., solo el 30 \% del mercado de bonos convertibles tiene grado de inversión (pero está en aumento). En Europa y Japón esta cifra es del 85%.

\underline{¿Por qué comprar un convertible?}
Los bonos convertibles pueden ser inversiones muy atractivas por las siguientes razones:
\begin{itemize}
    \item Participación al alza con protección a la baja (a menos que se produzca un incumplimiento).
    \item El cupón suele ser mayor que la rentabilidad por dividendo de la acción.
    \item Algunos inversores pueden tener prohibido participar directamente en el mercado de valores. La naturaleza de deuda del instrumento puede hacerlo atractivo.
    \item Legalmente, al ser títulos de deuda, tienen prioridad sobre la acción en caso de incumplimiento.
\end{itemize}

\underline{Algunas estadísticas}
Los bonos convertibles ya no son adquiridos por inversores privados; los fondos de cobertura han dominado este mercado desde hace tiempo.
\begin{itemize}
    \item A nivel mundial, la capitalización de los bonos convertibles es de 500 000 millones de dólares.
    \item 400 fondos de cobertura se centran en el arbitraje de bonos convertibles (de un total de 7000 fondos de cobertura).
    \item El arbitraje de bonos convertibles es la tercera mejor estrategia de fondos de cobertura (desde 1993), después de Equity Market Neutral y Event Driven, con un ratio de Sharpe de 1,04. (Nota: El ratio de Sharpe del S\&P 500 fue de 1,18 durante el mismo período).
    \item Los fondos de cobertura poseen el 70 \% de los bonos convertibles.
\end{itemize}
El arbitraje de bonos convertibles ha dejado de ser una estrategia rentable durante un tiempo. Tantos fondos de cobertura se han sumado a esta tendencia que la situación se ha desmoronado. Las posibilidades de arbitraje se han evaporado.



\subsection{Práctica de mercado}

Como hemos visto, el valor de un bono convertible (CB) es al menos su valor de conversión. El componente de bono del CB empuja el valor hacia arriba. Podemos calcular el \textbf{precio de conversión de mercado}, que es la cantidad que el tenedor del CB efectivamente paga por una acción si ejerce la opción de convertir inmediatamente:
\begin{equation*}
    \text{precio de conversión de mercado} = \frac{\text{precio de mercado del CB}}{\text{ratio de conversión}} \geq \text{precio del subyacente}
\end{equation*}

El comprador del CB paga una prima sobre el precio de mercado de la acción subyacente. Esta prima se mide mediante el \textbf{ratio de prima de conversión de mercado}:
\begin{equation*}
    \text{ratio de prima de conversión de mercado} = \frac{\text{precio de conversión de mercado} - \text{precio de mercado actual del CB}}{\text{precio de mercado de la acción subyacente}}
\end{equation*}

Al mantener la acción subyacente se reciben dividendos; al mantener el CB se reciben los cupones del bono pero no los dividendos de la acción. Una medida del beneficio de mantener el bono es el \textbf{diferencial favorable de ingresos}, medido por:
\begin{equation*}
    \text{diferencial favorable de ingresos} = \frac{\text{interés del cupón del bono} - (\text{ratio de conversión} \times \text{dividendo por acción})}{\text{ratio de conversión}}
\end{equation*}

Tras calcular esto, se puede estimar el \textbf{periodo de recuperación de la prima}, es decir, cuánto tiempo se tarda en recuperar la prima pagada sobre el precio de la acción:
\begin{equation*}
    \text{periodo de recuperación de la prima} = \frac{\text{precio de conversión de mercado} - \text{precio de mercado actual del CB}}{\text{diferencial favorable de ingresos}}
\end{equation*}

Esto no tiene en cuenta el valor temporal del dinero. Para valores bajos de la acción subyacente, la posibilidad de conversión es remota y el CB se comporta como un bono vainilla. Si el precio de la acción sube lo suficiente, el CB será convertido y comenzará a comportarse como la acción.




\subsection{Valoración de bonos convertibles con interés constante}
Sea $S$ el precio del subyacente, $T$ la fecha de vencimiento, $n$ el número del activo por el que se puede convertir el bono (i.e.\ $nS$) y $V=V(S,t)$ el valor del contrato. Se puede construir una cartera de un bono convertible y $-\Delta$ acciones del subyacente y se llega a la siguiente variación:
\begin{equation*}
    d\Pi = \frac{\partial V}{\partial t} dt + \frac{\partial V}{\partial S} dS + \frac{1}{2} \sigma^2 S^2 \frac{\partial^2 V}{\partial S^2} dt - \Delta dS
\end{equation*}
luego se elige $\Delta = \frac{\partial V}{\partial S}$ para eliminar el riesgo. Para evitar arbitraje, el retorno de esta cartera libre de riesgo no debe mayor que un deposito en el banco, por lo que:
\begin{equation*}
    \boxed{\frac{\partial V}{\partial t} + \frac{1}{2} \sigma^2 S^2 \frac{\partial^2 V}{\partial S^2} + \left(rS - D(S,t)\right) \frac{\partial V}{\partial S} - rV \leq 0}
\end{equation*}
El menor o igual no produce arbitraje por la característica de poder convertir el bono. Es decir, se podría vender la cartera e invertir el dinero en el banco obteniendo beneficio segurado, pero como se puede convertir el bono, ya no es asegurado el beneficio. El mayor o igual si que produciría arbitraje ya que se podría predir prestado con tipo de interés $r$, contruir la cartera y obtener una rentabilidad mayor que $r$.

Reescalando el pago principal a 1, la condición final es
\begin{equation*}
    \boxed{V(S,T)=1}
\end{equation*}

Si se pagan cupones discretos de contidad $K$, se tiene la condición de salto:
\begin{equation*}
    \boxed{V(S, t_c^-) = V(S, t_c^+) + K}
\end{equation*}

Ya que el bono se puede convertir en $n$ acciones, se tiene la condición:
\begin{equation*}
    \boxed{V \geq nS}
\end{equation*}

Se debe añadir también la \underline{condición de continuidad de $\partial V/\partial S$}.

Se puede ver que hay un problema en la compatibilidad de condiciones cuando $t \to T^-$ ya que se debe de cumplir que $V(S,T)=1$ y $V \geq nS$. Para solucionarlo, se impone la condición
\begin{equation*}
    \boxed{V(S,T^-) = \max(nS, 1)}
\end{equation*}

En ausencia de credit risk, se imponen además las condiciones:
\begin{equation*}
    \boxed{V(S, t) \sim nS, \quad S \to \infty}
\end{equation*}
y cuando $S=0$, el valor es la actualización del principal y de los cupones:
\begin{equation*}
    \boxed{V(0, t) = e^{-r(T-t)} + \sum K e^{-r(t_c-t)}}
\end{equation*}
donde se han sumado solo los cupones futuros. Estas dos últimas condiciones implican que para precios altos del subyacente, el bono se comporta como una acción y para precios bajos como un bono tradicional.

Si las características especiales del bono solo aplican a tiempos concretos, se deben aplicar sus correpondientes condiciones a esos tiempos. 


\subsubsection{Call and Put Features}
En ocasiones existe una call feature, que permite al emisor el derecho a recomprar el bono por una cantidad específica $M_C$ (que puede depender del tiempo). Es implica la condición:
\begin{equation*}
    \boxed{V(S,t) \leq M_C}
\end{equation*}

Una put feature es similar pero permite al comprador devolver el bono al emisor por una cantidad $M_P$ (que puede depender del tiempo). Esto implica la condición:
\begin{equation*}
    \boxed{V(S,t) \geq M_P}
\end{equation*}




\subsection{Valoración de bonos convertibles con tasas de interés estocásticas: modelos de dos factores}
En este caso se tiene que el bono convertible es de la forma
\begin{equation*}
    V = V(S,r,t)
\end{equation*}
donde se tienen los procesos estocásticos
\begin{align*}
    dS = a(S,t)\,dt + b(S,t)\,dX_1, \\[2ex]
    dr = u(r,t)\,dt + w(r,t)\,dX_2.
\end{align*}
donde los procesos de Wiener tiene una correlación:
\begin{equation*}
    \mathbb{E}[dX_1 dX_2] = \rho\,dt, \qquad -1\leq \rho(r,S,t) \leq 1.
\end{equation*}

Se construye la cartera
\begin{equation*}
    \Pi = V - \Delta_2 Z - \Delta_1 S
\end{equation*}
siendo $T_1$ el vencimiento del bono convertible y $T_2$ el vencimiento del bono de cupón cero.
Por el lema de Itô para procesos correlacionados~\eqref{eq:ItoCorrelated} se tiene que
\begin{equation*}
    dV = \left( \frac{\partial V}{\partial t} + \frac{1}{2} b^2 \frac{\partial^2 V}{\partial S^2} + \rho b w \frac{\partial^2 V}{\partial S \partial r} + \frac{1}{2} w^2 \frac{\partial^2 V}{\partial r^2} \right)\,dt + \frac{\partial V}{\partial S}\,dS + \frac{\partial V}{\partial r}\,dr
\end{equation*}
y la variación de un bono se ha visto en~\eqref{eq:variacion_bono} que es:
\begin{equation*}
    dZ = \left((w\lambda - u) \frac{\partial Z}{\partial r} + rZ\right)\,dt + \frac{\partial Z}{\partial r}\,dr
\end{equation*}

por lo que la variación de la cartera es
\begin{align*}
    d\Pi =& dV - \Delta_2 dZ - \Delta_1 dS \\
    =& \left( \frac{\partial V}{\partial t} + \frac{1}{2} b^2 \frac{\partial^2 V}{\partial S^2} + \rho b w \frac{\partial^2 V}{\partial S \partial r} + \frac{1}{2} w^2 \frac{\partial^2 V}{\partial r^2} \right)\,dt + \frac{\partial V}{\partial S}\,dS + \frac{\partial V}{\partial r}\,dr \\
    &- \Delta_2 \left(\left((w\lambda - u) \frac{\partial Z}{\partial r} + rZ\right)\,dt + \frac{\partial Z}{\partial r}\,dr \right) \\
    &- \Delta_1 dS \\
    =& \left( \frac{\partial V}{\partial t} + \frac{1}{2} b^2 \frac{\partial^2 V}{\partial S^2} + \rho b w \frac{\partial^2 V}{\partial S \partial r} + \frac{1}{2} w^2 \frac{\partial^2 V}{\partial r^2} - \Delta_2 \left((w\lambda - u) \frac{\partial Z}{\partial r} + rZ\right) \right)\,dt \\
    &+ \left( \frac{\partial V}{\partial S} - \Delta_1 \right)\,dS + \left( \frac{\partial V}{\partial r} - \Delta_2 \frac{\partial Z}{\partial r} \right)\,dr
\end{align*}
que eligiendo
\begin{align*}
    \Delta_1 &= \frac{\partial V}{\partial S} \\
    \Delta_2 &= \frac{\partial V/\partial r}{\partial Z/\partial r}
\end{align*}
se obtiene que
\begin{align*}
    d\Pi =& \left( \frac{\partial V}{\partial t} + \frac{1}{2} b^2 \frac{\partial^2 V}{\partial S^2} + \rho b w \frac{\partial^2 V}{\partial S \partial r} + \frac{1}{2} w^2 \frac{\partial^2 V}{\partial r^2} - \frac{\partial V/\partial r}{\partial Z/\partial r} \left((w\lambda - u) \frac{\partial Z}{\partial r} + rZ\right) \right)\,dt \\
\end{align*}

Para evitar arbitraje esta cartera libre de riesgo debe ser igual a
\begin{align*}
    r \Pi dt = r \left( V - \Delta_2 Z - \Delta_1 S \right) dt = r \left( V - \frac{\partial V/\partial r}{\partial Z/\partial r} Z - \frac{\partial V}{\partial S} S \right) dt
\end{align*}
es decir,
\begin{align*}
    &\frac{\partial V}{\partial t} + \frac{1}{2} b^2 \frac{\partial^2 V}{\partial S^2} + \rho b w \frac{\partial^2 V}{\partial S \partial r} + \frac{1}{2} w^2 \frac{\partial^2 V}{\partial r^2} - \frac{\partial V/\partial r}{\cancel{\partial Z/\partial r}} \left((w\lambda - u) \cancel{\frac{\partial Z}{\partial r}} + \cancel{rZ} \right) = r \left( V - \cancel{\frac{\partial V/\partial r}{\partial Z/\partial r} Z} - \frac{\partial V}{\partial S} S \right) \\[2ex]
    \implies & \boxed{\frac{\partial V}{\partial t} + \frac{1}{2} b^2 \frac{\partial^2 V}{\partial S^2} + \rho b w \frac{\partial^2 V}{\partial S \partial r} + \frac{1}{2} w^2 \frac{\partial^2 V}{\partial r^2} + rS \frac{\partial V}{\partial S} + (u - w\lambda )\frac{\partial V}{\partial r} - rV = 0}
\end{align*}
donde $\lambda(r,S,t)$ es el market price of interest rate risk.

Usando $a(S,t) = \mu S$ y $b(S,t) = \sigma S$ como se ha estado usando hasta ahora, se obtiene:
\begin{equation*}
    \boxed{\frac{\partial V}{\partial t} + \frac{1}{2} \sigma^2 S^2 \frac{\partial^2 V}{\partial S^2} + \rho \sigma S w \frac{\partial^2 V}{\partial S \partial r} + \frac{1}{2} w^2 \frac{\partial^2 V}{\partial r^2} + rS \frac{\partial V}{\partial S} + (u - w\lambda )\frac{\partial V}{\partial r} - rV = 0}
\end{equation*}

Las condiciones de contorno son similares a las expuestas en el apartado anterior.



\subsection{Dependencia del camino en bonos convertibles}

El bono convertible mencionado en el apartado de interés constante es el más simple, pero pueden ser mucho más complejos. Una fuente de complejidad es, como siempre, la dependencia de la trayectoria (\textit{path dependency}). Un bono convertible típico dependiente de la trayectoria sería el siguiente.

El bono paga \$1 al vencimiento, es decir, $t = T$. Antes del vencimiento, puede convertirse, en cualquier momento, por $n$ del subyacente. Inicialmente, $n$ se fija en una constante $n_b$. En el tiempo $T_0$ la ratio de conversión $n$ se fija en alguna función del subyacente en ese momento, $n_a(S(T_0))$. Restringiendo la atención a una tasa de interés determinista, este problema tridimensional satisface la misma ecuación que antes:
\begin{equation*}
    \frac{\partial V}{\partial t} + \frac{1}{2}\sigma^2 S^2 \frac{\partial^2 V}{\partial S^2} + rS \frac{\partial V}{\partial S} - rV \leq 0
\end{equation*}
Aquí $V(S, S, t)$ es el valor del bono con $S$ siendo el valor de $S$ en el tiempo $T_0$. Los dividendos y cupones se pueden añadir a esta ecuación según sea necesario. El bono convertible satisface la condición final
\begin{equation*}
    \boxed{V(S, S, T) = 1}
\end{equation*}
la restricción
\begin{equation*}
    \boxed{V(S, S, t) \geq n(S, S)}
\end{equation*}
donde
\begin{equation*}
    n(S, S) = 
    \begin{cases}
        n_b & \text{si } t \leq T_0 \\
        n_a(S) & \text{si } t > T_0
    \end{cases}
\end{equation*}
y la condición de salto
\begin{equation*}
    \boxed{V(S, S, T_0^-) = V(S, S, T_0^+)}
\end{equation*}

Este problema es tridimensional, tiene variables independientes $S$, $S$ y $t$. Se podría introducir tasas de interés estocásticas con poco trabajo teórico extra (aparte de elegir el modelo de tasas de interés), pero el tiempo de cómputo requerido para el problema resultante de cuatro dimensiones podría hacer inviable su resolución.





\subsection{Dilución en bonos convertibles}

En la práctica, la conversión de un bono en acciones requiere que la empresa emita $n$ nuevas acciones. Esto contrasta con las opciones, donde el ejercicio no altera el número de acciones existentes.

Redefinimos $S$ como el precio de la acción antes de la conversión y $N$ como el número de acciones existentes. El valor total de la empresa antes de la conversión es $NS - V$, donde $V$ es el valor del bono convertible. Tras la conversión, el precio de la acción pasa a ser:
\begin{equation*}
    \frac{NS - V}{N}
\end{equation*}
y no simplemente $S$. La restricción de que el valor del bono convertible debe ser mayor que el valor de conversión se expresa como:
\begin{equation*}
    V \geq n \frac{NS - V}{N}
\end{equation*}
que se puede reescribir como:
\begin{equation}
    \boxed{V \geq \frac{N}{n + N} nS}
    \label{eq:dilucion}
\end{equation}
Además, debe cumplirse:
\begin{equation}
    V \leq S
    \label{eq:cond_bancarrota_CB}
\end{equation}
y la condición final:
\begin{equation*}
    V(S, T) = 1
\end{equation*}

La ecuación~\eqref{eq:dilucion} acota el valor del bono por debajo, considerando la dilución provocada por la emisión de nuevas acciones. La restricción~\eqref{eq:cond_bancarrota_CB} permite a la empresa declararse en quiebra si el bono se vuelve demasiado valioso; también se puede ver como que no tiene sentido que el derecho a comprar una acción valga más que la propia acción. El factor $\frac{N}{n + N}$ se denomina \textbf{dilución}. En el límite $n/N \to 0$, recuperamos la condición $V \geq nS$.



593




