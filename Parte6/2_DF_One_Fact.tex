\section{Diferencias finitas para modelos de un factor}
Dados $S_\infty$ (normamente $3E$ o $4E$) y $T$, se plantea una malla de diferencias finitas del dominio acotado: dados los números naturales $N > 1$ y $M > 1$ se definen los pasos de tiempo y de activo de la malla
\begin{equation*}
    \Delta t = T/(N+1), \quad \Delta S = S_\infty/(M+1)
\end{equation*}
y los puntos (nodos) de la malla de diferencias finitas son
\begin{equation*}
    V_j^i = V(t_j, S_i) = (j\,\Delta t,\, i\,\Delta S), \quad j = 0, \ldots, N+1;\; i = 0, \ldots, M+1.
\end{equation*}

A continuación, se plantea el siguiente $\theta$-esquema:
\begin{align*}
    \frac{\partial V}{\partial t}\left(t_j, S_i\right) &\approx \frac{V_{j+1}^i - V_j^i}{\Delta t} \\
    \frac{\partial V}{\partial S}\left(t_j, S_i\right) &\approx \theta \frac{V_j^{i+1} - V_j^i}{\Delta S} + (1-\theta) \frac{V_{j+1}^{i+1} - V_{j+1}^i}{\Delta S} \\
    \frac{\partial^2 V}{\partial S^2}\left(t_j, S_i\right) &\approx \theta \frac{V_j^{i+1} - 2V_j^i + V_j^{i-1}}{\Delta S^2}  + (1-\theta) \frac{V_{j+1}^{i+1} - 2V_{j+1}^i + V_{j+1}^{i-1}}{\Delta S^2} \\
    V\left(t_j, S_i\right) &\approx \theta\, V_j^i + (1-\theta)\, V_{j+1}^i
\end{align*}

Se considera ahora la EDP parabólica general:
\begin{align*}
    &\frac{\partial V}{\partial t} + a(t,S) \frac{\partial^2 V}{\partial S^2} + b(t,S) \frac{\partial V}{\partial S} + c(t,S) V + f(t,S) = 0 \\
    \implies &\frac{V_{j+1}^i - V_j^i}{\Delta t} \\
    &+ a_j^i \left( \theta \frac{V_j^{i+1} - 2V_j^i + V_j^{i-1}}{\Delta S^2}  + (1-\theta) \frac{V_{j+1}^{i+1} - 2V_{j+1}^i + V_{j+1}^{i-1}}{\Delta S^2} \right) \\
    &+ b_j^i \left( \theta \frac{V_j^{i+1} - V_j^i}{\Delta S} + (1-\theta) \frac{V_{j+1}^{i+1} - V_{j+1}^i}{\Delta S} \right)\\
    &+ c_j^i \left( \theta V_j^i + (1-\theta) V_{j+1}^i \right) \\
    &+ f_j^i = 0 \\
\end{align*}
que agrupando por V's:
\begin{align*}
    &\left( \frac{\theta}{\Delta S^2} a_j^i \right)V_j^{i-1} \\
    &+\left( -\frac{1}{\Delta t} - \frac{2\theta}{\Delta S^2}a_j^i  - \frac{\theta}{\Delta S}b_j^i + \theta c_j^i \right)V_j^i \\
    &+\left( \frac{\theta}{\Delta S^2} a_j^i + \frac{\theta}{\Delta S}b_j^i \right)V_j^{i+1} \\
    &+\left( \frac{1-\theta}{\Delta S^2} a_j^i \right)V_{j+1}^{i-1} \\
    &+\left( \frac{1}{\Delta t} - \frac{2(1-\theta)}{\Delta S^2}a_j^i - \frac{(1-\theta)}{\Delta S}b_j^i + (1-\theta) c_j^i \right)V_{j+1}^i \\
    &+\left( \frac{(1-\theta)}{\Delta S^2} a_j^i + \frac{(1-\theta)}{\Delta S}b_j^i \right)V_{j+1}^{i+1} \\
    &+ f_j^i = 0 \\
\end{align*}
Se define
\begin{equation*}
    \alpha = \frac{1}{\Delta S^2}, \qquad \beta = \frac{1}{\Delta S}, \qquad \gamma = \frac{1}{\Delta t}
\end{equation*}
por lo que
\begin{align*}
    &\theta\left( \alpha a_j^i \right)V_j^{i-1} +\left( -\gamma - \theta (2\alpha a_j^i  + \beta b_j^i - c_j^i) \right)V_j^i +\theta\left( \alpha a_j^i + \beta b_j^i \right)V_j^{i+1} \\
    &+(1-\theta)\left( \alpha a_j^i \right)V_{j+1}^{i-1} +\left( \gamma - (1-\theta) (2\alpha a_j^i + \beta b_j^i - c_j^i) \right)V_{j+1}^i \\
    &+(1-\theta)\left( \alpha a_j^i + \beta b_j^i \right)V_{j+1}^{i+1} + f_j^i = 0 \\
\end{align*}
y dado un $j$, se considera:
\begin{equation*}
    \eta_i = \alpha a_j^i, \qquad \varphi_i = 2\alpha a_j^i  + \beta b_j^i - c_j^i, \qquad \psi_i = \alpha a_j^i + \beta b_j^i
\end{equation*}
por lo que
\begin{align*}
    &\theta\eta_i V_j^{i-1} +\left( -\gamma - \theta \varphi_i \right)V_j^i +\theta\psi_i V_j^{i+1} \\
    &+(1-\theta)\eta_i V_{j+1}^{i-1} +\left( \gamma - (1-\theta) \varphi_i \right)V_{j+1}^i +(1-\theta)\psi_i V_{j+1}^{i+1} + f_j^i = 0 \\
\end{align*}
es decir, para cada $j$ se debe resolver el sistema de ecuaciones:
\begin{align}
    &\bigg[ \theta\eta_i \bigg] V_j^{i-1} + \bigg[ -\gamma - \theta \varphi_i \bigg] V_j^i + \bigg[ \theta\left( \psi_i \right) \bigg] V_j^{i+1} \nonumber \\
    &= \bigg[ - (1-\theta)\eta_i \bigg] V_{j+1}^{i-1} + \bigg[ - \gamma + (1-\theta) \varphi_i \bigg] V_{j+1}^i + \bigg[ - (1-\theta)\psi_i \bigg] V_{j+1}^{i+1} + \bigg[ - f_j^i \bigg] \label{eq:DF_theta_general}
\end{align}
Claramente, existe un problema a la hora de usar los valores en los bordes, $V_j^0$ y $V_j^{M+1}$, que no se conocen. Existen varias maneras de afrontar este problema.



\subsection{Condiciones frontera conocidas}
Si se conocen las condiciones frontera (en los límites de $S$, que generalmente son $S=0$ y $S\to\infty$), entonces se pueden imponer directamente. En este caso, para cada instante de tiempo $j=N, N-1, \ldots, 0$ se deben resolver el siguiente sistema de ecuaciones:
\begin{equation*}
    \boxed{A V_j = B V_{j+1} - f_j + C}
\end{equation*}
donde las matrices tridiagonales $A$ y $B$ son:
\begin{align*}
    &\boxed{
        A = \begin{pmatrix}
            A_{1,1} & A_{1,2} & 0 & \cdots & 0 & 0 & 0 \\
            A_{2,1} & A_{2,2} & A_{2,3} & \cdots & 0 & 0 & 0 \\
            0 & A_{3,2} & A_{3,3} & \cdots & 0 & 0 & 0 \\
            \vdots & \vdots & \vdots & \ddots & \vdots & \vdots & \vdots \\
            0 & 0 & 0 & \cdots & A_{M-2,M-2} & A_{M-2,M-1} & 0 \\
            0 & 0 & 0 & \cdots & A_{M-1,M-2} & A_{M-1,M-1} & A_{M-1,M} \\
            0 & 0 & 0 & \cdots & 0  & A_{M,M-1} & A_{M,M}
        \end{pmatrix},
        \quad \begin{cases}
            A_{i,i-1} = \theta\eta_i \\
            A_{i,i} = -\gamma - \theta \varphi_i \\
            A_{i,i+1} = \theta\psi_i \\
        \end{cases}
    } \\
    & \boxed{
        B = \begin{pmatrix}
            B_{1,1} & B_{1,2} & 0 & \cdots & 0 & 0 & 0 \\
            B_{2,1} & B_{2,2} & B_{2,3} & \cdots & 0 & 0 & 0 \\
            0 & B_{3,2} & B_{3,3} & \cdots & 0 & 0 & 0 \\
            \vdots & \vdots & \vdots & \ddots & \vdots & \vdots & \vdots \\
            0 & 0 & 0 & \cdots & B_{M-2,M-2} & B_{M-2,M-1} & 0 \\
            0 & 0 & 0 & \cdots & B_{M-1,M-2} & B_{M-1,M-1} & B_{M-1,M} \\
            0 & 0 & 0 & \cdots & 0  & B_{M,M-1} & B_{M,M}
        \end{pmatrix}
        \quad \begin{cases}
            B_{i,i-1} = -(1-\theta)\eta_i \\
            B_{i,i} = -\gamma + (1-\theta) \varphi_i \\
            B_{i,i+1} = -(1-\theta)\psi_i \\
        \end{cases}
    }
\end{align*}
y donde
\begin{align*}
    &\boxed{
        V_j = \begin{pmatrix}
            V_j^1 \\
            V_j^2 \\
            \vdots \\
            V_j^M
        \end{pmatrix}, \quad
        V_{j+1} = \begin{pmatrix}
            V_{j+1}^1 \\
            V_{j+1}^2 \\
            \vdots \\
            V_{j+1}^M
        \end{pmatrix}, \quad
        f_j = \begin{pmatrix}
            f_j^1 \\
            f_j^2 \\
            \vdots \\
            f_j^M
        \end{pmatrix}
    } \\
    &\boxed{
        C = \begin{pmatrix}
            -\theta\eta_1 V_j^{0} - (1-\theta)\eta_1 V_{j+1}^{0} \\
            0 \\
            \vdots \\
            -\theta\psi_M V_j^{M+1} - (1-\theta)\psi_M V_{j+1}^{M+1} \\
        \end{pmatrix}
        = \begin{pmatrix}
            -A_{1,0}V_j^{0} + B_{1,0}V_{j+1}^{0} \\
            0 \\
            \vdots \\
            -A_{M, M+1} V_j^{M+1} + B_{M, M+1} V_{j+1}^{M+1} \\
        \end{pmatrix}
    }
\end{align*}

En ocasiones, sencillo conocer el valor de la opción en $S=0$. Generalmente, dicho límite es sencillo conocerlo cuando en la EDP se cumple que $a(t,0)=0$ y $b(t,0)=0$ (en la mayoría de casos, $a(t,S)=1/2\sigma^2S^2$ y $b(t,S)=rS$). En este caso, la EDP en $S=0$ se puede reducir a la EDO:\@
\begin{align*}
    &\frac{\partial V}{\partial t} + c(t, S) V + f(t, S) = 0 \\
\end{align*}
cuya solucion teniendo en cuenta un payoff $V(T,S)=\phi(S)$ es
\begin{equation*}
    \boxed{V(t, 0) = \phi(0) e^{\int_t^T c(u, 0)\,du} + \int_t^T f(u, 0) e^{\int_u^T c(v, 0)\,dv}\,du}
\end{equation*}
de igual manera, suele ser común que $f(t,S)=0$ y que $c(t,S)=-r$, por lo que la solución se reduce a:
\begin{equation*}
    V(t, 0) = \phi(0) e^{-r(T-t)}
\end{equation*}

Por otro lado, el valor en $S\to\infty$ se puede obtener asumiendo que en $S\to\infty$ el valor de la opción se comporta de manera lineal, es decir, que existe $A(t), B(t)$ tales que
\begin{equation*}
    V(t, S) = A(t) S + B(t), \quad S \to \infty
\end{equation*}
haciendo dicha suposición, sutituyendo en la EDP una vez se conozcan los términos de la EDP y usando el payoff cuando $S\to\infty$, se podrían obtener $A(t), B(t)$.




\subsection{Condición de frontera conocida solo en el límite inferior}\label{sec:DF_condicion_frontera_inferior_conocida}
En este caso el problema radica en los valores de $V_j^{M+1}$.

Asumiendo que cuando $S\to\infty$ se comporta de manera lineal tal y como se ha comentado en la sección anterior:
\begin{equation*}
    \boxed{V(t, S) = A(t) S + B(t), \quad S \to \infty}
\end{equation*}
Entonces se sabe que la derivada segunda respecto al activo es nula. Por lo tanto, usando diferencias finitas hacia atrás en la segunda derivada:
\begin{align*}
    &0 = \frac{\partial^2 V}{\partial S^2} = \frac{V_j^{M+1} - 2V_j^{M} + V_j^{M-1}}{\Delta S^2} + O(\Delta S^2) \\
    \implies &V_j^{M+1} = 2V_j^{M} - V_j^{M-1}
\end{align*}
y sustituyendo en la ecuación~\eqref{eq:DF_theta_general} para $i=M$:
\begin{align*}
    &\theta\eta_M V_j^{M-1} +\left( -\gamma - \theta \varphi_M \right)V_j^M +\theta\psi_M V_j^{M+1} \\
    &= - (1-\theta)\eta_M V_{j+1}^{M-1} -\left( \gamma - (1-\theta) \varphi_M \right)V_{j+1}^M - (1-\theta)\psi_M V_{j+1}^{M+1} - f_j^M \\
    \implies &\theta\eta_M V_j^{M-1} +\left( -\gamma - \theta \varphi_M \right)V_j^M +\theta\psi_M \left(2V_j^{M} - V_j^{M-1}\right) \\
    &= - (1-\theta)\eta_M V_{j+1}^{M-1} -\left( \gamma - (1-\theta) \varphi_M \right)V_{j+1}^M - (1-\theta)\psi_M \left(2V_{j+1}^{M} - V_{j+1}^{M-1}\right) - f_j^M \\
    \implies &\bigg[ \theta(\eta_M - \psi_M) \bigg] V_j^{M-1} + \bigg[ -\gamma - \theta (\varphi_M - 2\psi_M) \bigg] V_j^M\\
    &= \bigg[ - (1-\theta)(\eta_M - \psi_M) \bigg] V_{j+1}^{M-1} + \bigg[ - \gamma + (1-\theta) (\varphi_M - 2\psi_M) \bigg] V_{j+1}^M + \bigg[ - f_j^M \bigg]
\end{align*}
por lo que el sistema a resolver es:
\begin{equation*}
    \boxed{A V_j = B V_{j+1} - f_j + C}
\end{equation*}
donde las matrices tridiagonales $A$ y $B$ son:
\begin{align*}
    &\boxed{
        A = \begin{pmatrix}
            A_{1,1} & A_{1,2} & 0 & \cdots & 0 & 0 & 0 \\
            A_{2,1} & A_{2,2} & A_{2,3} & \cdots & 0 & 0 & 0 \\
            0 & A_{3,2} & A_{3,3} & \cdots & 0 & 0 & 0 \\
            \vdots & \vdots & \vdots & \ddots & \vdots & \vdots & \vdots \\
            0 & 0 & 0 & \cdots & A_{M-2,M-2} & A_{M-2,M-1} & 0 \\
            0 & 0 & 0 & \cdots & A_{M-1,M-2} & A_{M-1,M-1} & A_{M-1,M} \\
            0 & 0 & 0 & \cdots & 0 & A_{M,M-1}^* & A_{M,M}^*
        \end{pmatrix},
        \quad \begin{cases}
            A_{i,i-1} = \theta\eta_i \\
            A_{i,i} = -\gamma - \theta \varphi_i \\
            A_{i,i+1} = \theta\psi_i \\
            A_{M,M-1}^* = \theta(\eta_M - \psi_M)\\
            A_{M,M}^* = -\gamma - \theta (\varphi_M - 2\psi_M)\\
        \end{cases}
    } \\
    & \boxed{
        B = \begin{pmatrix}
            B_{1,1} & B_{1,2} & 0 & \cdots & 0 & 0 & 0 \\
            B_{2,1} & B_{2,2} & B_{2,3} & \cdots & 0 & 0 & 0 \\
            0 & B_{3,2} & B_{3,3} & \cdots & 0 & 0 & 0 \\
            \vdots & \vdots & \vdots & \ddots & \vdots & \vdots & \vdots \\
            0 & 0 & 0 & \cdots & B_{M-2,M-2} & B_{M-2,M-1} & 0 \\
            0 & 0 & 0 & \cdots & B_{M-1,M-2} & B_{M-1,M-1} & B_{M-1,M} \\
            0 & 0 & 0 & \cdots & 0 & B_{M,M-1}^* & B_{M,M}^*
        \end{pmatrix}
        \quad \begin{cases}
            B_{i,i-1} = -(1-\theta)\eta_i \\
            B_{i,i} = -\gamma + (1-\theta) \varphi_i \\
            B_{i,i+1} = -(1-\theta)\psi_i \\
            B_{M,M-1}^* = -(1-\theta)(\eta_M - \psi_M)\\
            B_{M,M}^* = -\gamma + (1-\theta) (\varphi_M - 2\psi_M)\\
        \end{cases}
    }
\end{align*}
y donde
\begin{align*}
    &\boxed{
        V_j = \begin{pmatrix}
            V_j^1 \\
            V_j^2 \\
            \vdots \\
            V_j^M
        \end{pmatrix}, \quad
        V_{j+1} = \begin{pmatrix}
            V_{j+1}^1 \\
            V_{j+1}^2 \\
            \vdots \\
            V_{j+1}^M
        \end{pmatrix}, \quad
        f_j = \begin{pmatrix}
            f_j^1 \\
            f_j^2 \\
            \vdots \\
            f_j^M
        \end{pmatrix}
    } \\
    &\boxed{
        C = \begin{pmatrix}
            -\theta\eta_1 V_j^{0} - (1-\theta)\eta_1 V_{j+1}^{0} \\
            0 \\
            \vdots \\
            0 \\
        \end{pmatrix}
        = \begin{pmatrix}
            -A_{1,0}V_j^{0} + B_{1,0}V_{j+1}^{0} \\
            0 \\
            \vdots \\
            0 \\
        \end{pmatrix}
    }
\end{align*}
y teniendo en cuenta que en cada iteración temporal se debe calcular:
\begin{equation*}
    \boxed{V_j^{M+1} = 2V_j^{M} - V_j^{M-1}}
\end{equation*}
para cada instante de tiempo.





\subsection{Condición de frontera conocida solo en el límite superior}\label{sec:DF_condicion_frontera_superior_conocida}
En este caso el problema radica en los valores de $V_j^0$.

Como se ha comentado anteriormente, es común que en $S=0$ se cumpla $a(t,0)=0$ y $b(t,0)=0$ (en la mayoría de casos, $a(t,S)=1/2\sigma^2S^2$ y $b(t,S)=rS$). En este caso, la EDP en $S=0$ se puede reducir a la EDO:\@
\begin{align*}
    &\frac{\partial V}{\partial t} + c(t, S) V + f(t, S) = 0 \\
\end{align*}
que sustituyendo con las fórmulas de diferencias finitas usandas en $i=0$ sería:
\begin{align*}
    &\frac{V_{j+1}^0 - V_j^0}{\Delta t} + c_j^0 \left( \theta V_j^0 + (1-\theta) V_{j+1}^0 \right) + f_j^0 = 0 \\
    \implies &V_{j+1}^0 - V_j^0 + c_j^0 \Delta t \theta V_j^0 + c_j^0 \Delta t (1-\theta) V_{j+1}^0 + \Delta t f_j^0 = 0 \\
    \implies &\bigg[ 1 - c_j^0 \Delta t \theta \bigg] V_j^0 = \bigg[ 1 + c_j^0 \Delta t (1-\theta) \bigg] V_{j+1}^0 + \Delta t f_j^0 \\
    \implies &V_j^0 = \frac{1 + c_j^0 \Delta t (1-\theta)}{1 - c_j^0 \Delta t \theta} V_{j+1}^0 + \frac{\Delta t}{1 - c_j^0 \Delta t \theta} f_j^0
\end{align*}
y sustituyendo en la ecuación~\eqref{eq:DF_theta_general} para $i=1$:
\begin{align*}
    &\theta\eta_1 V_j^{0} +\left( -\gamma - \theta \varphi_1 \right)V_j^1 +\theta\psi_1 V_j^{2} \\
    &= - (1-\theta)\eta_1 V_{j+1}^{0} -\left( \gamma - (1-\theta) \varphi_1 \right)V_{j+1}^1 - (1-\theta)\psi_1 V_{j+1}^{2} - f_j^1 \\
    \implies &\theta\eta_1 \left( \frac{1 + c_j^0 \Delta t (1-\theta)}{1 - c_j^0 \Delta t \theta} V_{j+1}^0 + \frac{\Delta t}{1 - c_j^0 \Delta t \theta} f_j^0 \right) +\left( -\gamma - \theta \varphi_1 \right)V_j^1 +\theta\psi_1 V_j^{2} \\
    &= - (1-\theta)\eta_1 V_{j+1}^{0} -\left( \gamma - (1-\theta) \varphi_1 \right)V_{j+1}^1 - (1-\theta)\psi_1 V_{j+1}^{2} - f_j^1 \\
    \implies &\bigg[ -\gamma - \theta \varphi_1 \bigg]V_j^1 + \bigg[ \theta \psi_1 \bigg]V_j^{2} \\
    &= \bigg[ - \gamma + (1-\theta) \varphi_1 \bigg]V_{j+1}^1 + \bigg[ - (1-\theta)\psi_1 \bigg]V_{j+1}^{2} \\
    &+ \bigg[- f_j^1 \bigg] + \bigg[ \eta_1 \left( \theta\frac{\Delta t}{1 - c_j^0 \Delta t \theta} f_j^0 + \left( \theta \frac{1 + c_j^0 \Delta t (1-\theta)}{1 - c_j^0 \Delta t \theta} - (1-\theta) \right) V_{j+1}^{0} \right) \bigg] \\
\end{align*}
por lo que el sistema a resolver es:
\begin{equation*}
    \boxed{A V_j = B V_{j+1} - f_j + C}
\end{equation*}
donde las matrices tridiagonales $A$ y $B$ son:
\begin{align*}
    &\boxed{
        A = \begin{pmatrix}
            A_{1,1} & A_{1,2} & 0 & \cdots & 0 & 0 & 0 \\
            A_{2,1} & A_{2,2} & A_{2,3} & \cdots & 0 & 0 & 0 \\
            0 & A_{3,2} & A_{3,3} & \cdots & 0 & 0 & 0 \\
            \vdots & \vdots & \vdots & \ddots & \vdots & \vdots & \vdots \\
            0 & 0 & 0 & \cdots & A_{M-2,M-2} & A_{M-2,M-1} & 0 \\
            0 & 0 & 0 & \cdots & A_{M-1,M-2} & A_{M-1,M-1} & A_{M-1,M} \\
            0 & 0 & 0 & \cdots & 0  & A_{M,M-1} & A_{M,M}
        \end{pmatrix},
        \quad \begin{cases}
            A_{i,i-1} = \theta\eta_i \\
            A_{i,i} = -\gamma - \theta \varphi_i \\
            A_{i,i+1} = \theta\psi_i \\
        \end{cases}
    } \\
    & \boxed{
        B = \begin{pmatrix}
            B_{1,1} & B_{1,2} & 0 & \cdots & 0 & 0 & 0 \\
            B_{2,1} & B_{2,2} & B_{2,3} & \cdots & 0 & 0 & 0 \\
            0 & B_{3,2} & B_{3,3} & \cdots & 0 & 0 & 0 \\
            \vdots & \vdots & \vdots & \ddots & \vdots & \vdots & \vdots \\
            0 & 0 & 0 & \cdots & B_{M-2,M-2} & B_{M-2,M-1} & 0 \\
            0 & 0 & 0 & \cdots & B_{M-1,M-2} & B_{M-1,M-1} & B_{M-1,M} \\
            0 & 0 & 0 & \cdots & 0  & B_{M,M-1} & B_{M,M}
        \end{pmatrix}
        \quad \begin{cases}
            B_{i,i-1} = -(1-\theta)\eta_i \\
            B_{i,i} = -\gamma + (1-\theta) \varphi_i \\
            B_{i,i+1} = -(1-\theta)\psi_i \\
        \end{cases}
    }
\end{align*}
y donde
\begin{align*}
    &\boxed{
        V_j = \begin{pmatrix}
            V_j^1 \\
            V_j^2 \\
            \vdots \\
            V_j^M
        \end{pmatrix}, \quad
        V_{j+1} = \begin{pmatrix}
            V_{j+1}^1 \\
            V_{j+1}^2 \\
            \vdots \\
            V_{j+1}^M
        \end{pmatrix}, \quad
        f_j = \begin{pmatrix}
            f_j^1 \\
            f_j^2 \\
            \vdots \\
            f_j^M
        \end{pmatrix}
    } \\
    &\boxed{
        C = \begin{pmatrix}
            \eta_1 \left( \theta\frac{\Delta t}{1 - c_j^0 \Delta t \theta} f_j^0 + \left( \theta \frac{1 + c_j^0 \Delta t (1-\theta)}{1 - c_j^0 \Delta t \theta} - (1-\theta) \right) V_{j+1}^{0} \right) \\
            0 \\
            \vdots \\
            -\theta\psi_M V_j^{M+1} - (1-\theta)\psi_M V_{j+1}^{M+1} \\
        \end{pmatrix}
    }
\end{align*}
y teniendo en cuenta que en cada iteración temporal se debe calcular:
\begin{equation*}
    \boxed{V_j^0 = \frac{1 + c_j^0 \Delta t (1-\theta)}{1 - c_j^0 \Delta t \theta} V_{j+1}^0 + \frac{\Delta t}{1 - c_j^0 \Delta t \theta} f_j^0}
\end{equation*}






\subsection{Condiciones frontera desconocidas}
En este caso, los valores en los bordes $V_j^0$ y $V_j^{M+1}$ no se conocen. Por lo tanto, se deben imponer las aproximaciones descritas en las secciones~\ref{sec:DF_condicion_frontera_inferior_conocida} y~\ref{sec:DF_condicion_frontera_superior_conocida}:
\begin{align*}
    &V_j^0 = \frac{1 + c_j^0 \Delta t (1-\theta)}{1 - c_j^0 \Delta t \theta} V_{j+1}^0 + \frac{\Delta t}{1 - c_j^0 \Delta t \theta} f_j^0 \\
    &V_j^{M+1} = 2V_j^{M} - V_j^{M-1}
\end{align*}
lo que da una combinación de ambos casos anteriores. El sistema a resolver es:
\begin{equation*}
    \boxed{A V_j = B V_{j+1} - f_j + C}
\end{equation*}
donde las matrices tridiagonales $A$ y $B$ son:
\begin{align*}
    &\boxed{
        A = \begin{pmatrix}
            A_{1,1} & A_{1,2} & 0 & \cdots & 0 & 0 & 0 \\
            A_{2,1} & A_{2,2} & A_{2,3} & \cdots & 0 & 0 & 0 \\
            0 & A_{3,2} & A_{3,3} & \cdots & 0 & 0 & 0 \\
            \vdots & \vdots & \vdots & \ddots & \vdots & \vdots & \vdots \\
            0 & 0 & 0 & \cdots & A_{M-2,M-2} & A_{M-2,M-1} & 0 \\
            0 & 0 & 0 & \cdots & A_{M-1,M-2} & A_{M-1,M-1} & A_{M-1,M} \\
            0 & 0 & 0 & \cdots & 0  & A_{M,M-1}^* & A_{M,M}^*
        \end{pmatrix},
        \quad \begin{cases}
            A_{i,i-1} = \theta\eta_i \\
            A_{i,i} = -\gamma - \theta \varphi_i \\
            A_{i,i+1} = \theta\psi_i \\
            A_{M,M-1}^* = \theta(\eta_M - \psi_M)\\
            A_{M,M}^* = -\gamma - \theta (\varphi_M - 2\psi_M)\\
        \end{cases}
    } \\
    & \boxed{
        B = \begin{pmatrix}
            B_{1,1} & B_{1,2} & 0 & \cdots & 0 & 0 & 0 \\
            B_{2,1} & B_{2,2} & B_{2,3} & \cdots & 0 & 0 & 0 \\
            0 & B_{3,2} & B_{3,3} & \cdots & 0 & 0 & 0 \\
            \vdots & \vdots & \vdots & \ddots & \vdots & \vdots & \vdots \\
            0 & 0 & 0 & \cdots & B_{M-2,M-2} & B_{M-2,M-1} & 0 \\
            0 & 0 & 0 & \cdots & B_{M-1,M-2} & B_{M-1,M-1} & B_{M-1,M} \\
            0 & 0 & 0 & \cdots & 0  & B_{M,M-1}^* & B_{M,M}^*
        \end{pmatrix}
        \quad \begin{cases}
            B_{i,i-1} = -(1-\theta)\eta_i\\
            B_{i,i} = -\gamma + (1-\theta) \varphi_i\\
            B_{i,i+1} = -(1-\theta)\psi_i\\
            B_{M,M-1}^* = -(1-\theta)(\eta_M - \psi_M)\\
            B_{M,M}^* = -\gamma + (1-\theta) (\varphi_M - 2\psi_M)\\
        \end{cases}
    }
\end{align*}
y donde
\begin{align*}
    &\boxed{
        V_j = \begin{pmatrix}
            V_j^1 \\
            V_j^2 \\
            \vdots \\
            V_j^M
        \end{pmatrix}, \quad
        V_{j+1} = \begin{pmatrix}
            V_{j+1}^1 \\
            V_{j+1}^2 \\
            \vdots \\
            V_{j+1}^M
        \end{pmatrix}, \quad
        f_j = \begin{pmatrix}
            f_j^1 \\
            f_j^2 \\
            \vdots \\
            f_j^M
        \end{pmatrix}
    } \\
    &\boxed{
        C = \begin{pmatrix}
            -\theta\eta_1 V_j^{0} - (1-\theta)\eta_1 V_{j+1}^{0} + \eta_1 \left( \theta\frac{\Delta t}{1 - c_j^0 \Delta t \theta} f_j^0 + \left( \theta \frac{1 + c_j^0 \Delta t (1-\theta)}{1 - c_j^0 \Delta t \theta} - (1-\theta) \right) V_{j+1}^{0} \right) \\
            0 \\
            \vdots \\
            0 \\
        \end{pmatrix}
    }
\end{align*}
y teniendo en cuenta que en cada iteración temporal se debe calcular:
\begin{align*}
    &\boxed{V_j^0 = \frac{1 + c_j^0 \Delta t (1-\theta)}{1 - c_j^0 \Delta t \theta} V_{j+1}^0 + \frac{\Delta t}{1 - c_j^0 \Delta t \theta} f_j^0} \\
    &\boxed{V_j^{M+1} = 2V_j^{M} - V_j^{M-1}}
\end{align*}











\newpage




Estos valores se pueden imponer obteniendo las condiciones frontera de forma analítica o bien aproximándolas numéricamente. Para ello se van a hacer alguna suposiciones. En primer lugar, para $S=0$, es común que se cumpla $a(t, 0)=0$, $b(t, 0)=0$ (en el caso de caminos lognormales es común que $a(t, S)=1/2\sigma^2S^2$ y $b(t,S)=rS$). Por lo tanto se tendría que en $S=0$:
\begin{align*}
    &\frac{\partial V}{\partial t} + c(t, S) V + f(t, S) = 0 \\
\end{align*}
cuya solucion teniendo en cuenta que $V(T,S)=\phi(S)$ es
\begin{equation*}
    V(t, S) = \phi(S) e^{\int_t^T c(u, S)\,du} + \int_t^T f(u, S) e^{\int_u^T c(v, S)\,dv}\,du
\end{equation*}
si bien a no ser que $c, f$ sean constantes, suele ser más rápido aproximar usando directamente la EDO y métodos como Runge-Kutta, a no ser que se quiera una mayor precisión, en cuyo caso es más recomendable usar la solución exacta (se puede aproximar las integrales con métodos numéricos). Por lo tanto, los valores $V_j^0$ se puede asumir concidos. Falta conocer los valores $V_j^{M+1}$. Para ello se suele asumir que en $S_\infty$ el valor de la opción se comporta de manera lineal, es decir, que existe $A(t), B(t)$ tales que
\begin{equation*}
    V(t, S) = A(t) S + B(t), \quad S \to \infty
\end{equation*}
haciendo dicha suposición se podría obtener $A(t), B(t)$ si se conoce los términos de la EDP.\@ Otra manera de aproxiar el valor en $S_\infty$ es usando diferencias finitas en la segunda derivada. Usando diferencias hacia atrás:
\begin{align*}
    &0 = \frac{\partial^2 V}{\partial S^2} = \frac{V_j^{M+1} - 2V_j^{M} + V_j^{M-1}}{\Delta S^2} + O(\Delta S) \\
    \implies &V_j^{M+1} = 2V_j^{M} - V_j^{M-1}
\end{align*}

Recapitulando, sabiendo que para cada paso de tiempo $j$ se debe resolver:
\begin{align*}
    &\theta\eta_i V_j^{i-1} +\left( -\gamma - \theta \varphi_i \right)V_j^i +\theta\left( \psi_i \right)V_j^{i+1} \\
    &= - (1-\theta)\eta_i V_{j+1}^{i-1} -\left( \gamma - (1-\theta) \varphi_i \right)V_{j+1}^i - (1-\theta)\left( \psi_i \right)V_{j+1}^{i+1} - f_j^i
\end{align*}
sabiendo que para calcular los valores de $V$ en el borde $S=0$:
\begin{equation*}
    V_j^0 = V(t_j, 0) = \phi(0) e^{\int_{t_j}^T c(u, 0)\,du} + \int_{t_j}^T f(u, 0) e^{\int_u^T c(v, 0)\,dv}\,du
\end{equation*}
entonces se deben resolver las siguientes ecuaciones:
\begin{itemize}
    \item Para $i=1$:
        \begin{align*}
            &\theta\eta_1 V_j^{0} +\left( -\gamma - \theta \varphi_1 \right)V_j^1 +\theta\left( \psi_1 \right)V_j^{2} \\
            &= - (1-\theta)\eta_1 V_{j+1}^{0} -\left( \gamma - (1-\theta) \varphi_1 \right)V_{j+1}^1 - (1-\theta)\left( \psi_1 \right)V_{j+1}^{2} - f_j^1 \\
            \implies &\bigg[ -\gamma - \theta \varphi_1 \bigg]V_j^1 + \bigg[ \theta \psi_1 \bigg]V_j^{2} \\
            &= \bigg[ - \gamma + (1-\theta) \varphi_1 \bigg]V_{j+1}^1 + \bigg[ - (1-\theta)\psi_1 \bigg]V_{j+1}^{2} + \bigg[ -\theta\eta_1 V_j^{0} - (1-\theta)\eta_1 V_{j+1}^{0} - f_j^1 \bigg]
        \end{align*}

    \item Para $i=2, \ldots, M-1$:
    \begin{align*}
        &\bigg[ \theta\eta_i \bigg] V_j^{i-1} + \bigg[ -\gamma - \theta \varphi_i \bigg] V_j^i + \bigg[ \theta \psi_i \bigg] V_j^{i+1} \\
        &= \bigg[ - (1-\theta)\eta_i \bigg] V_{j+1}^{i-1} + \bigg[ - \gamma + (1-\theta) \varphi_i \bigg] V_{j+1}^i + \bigg[ - (1-\theta)\psi_i \bigg] V_{j+1}^{i+1} + \bigg[ - f_j^i \bigg]
    \end{align*}

    \item Para $i=M$:
    \begin{align*}
        &\theta\eta_M V_j^{M-1} +\left( -\gamma - \theta \varphi_M \right)V_j^M +\theta\psi_M V_j^{M+1} \\
        &= - (1-\theta)\eta_M V_{j+1}^{M-1} -\left( \gamma - (1-\theta) \varphi_M \right)V_{j+1}^M - (1-\theta)\psi_M V_{j+1}^{M+1} - f_j^M \\
        \implies &\theta\eta_M V_j^{M-1} +\left( -\gamma - \theta \varphi_M \right)V_j^M +\theta\psi_M \left(2V_j^{M} - V_j^{M-1}\right) \\
        &= - (1-\theta)\eta_M V_{j+1}^{M-1} -\left( \gamma - (1-\theta) \varphi_M \right)V_{j+1}^M - (1-\theta)\psi_M \left(2V_{j+1}^{M} - V_{j+1}^{M-1}\right) - f_j^M \\
        \implies &\bigg[ \theta(\eta_M - \psi_M) \bigg] V_j^{M-1} + \bigg[ -\gamma - \theta (\varphi_M - 2\psi_M) \bigg] V_j^M\\
        &= \bigg[ - (1-\theta)(\eta_M - \psi_M) \bigg] V_{j+1}^{M-1} + \bigg[ - \gamma + (1-\theta) (\varphi_M - 2\psi_M) \bigg] V_{j+1}^M + \bigg[ - f_j^M \bigg]
    \end{align*}
\end{itemize}


Por lo tanto, dada una EDP
\begin{equation*}
    \boxed{\frac{\partial V}{\partial t} + a(t,S) \frac{\partial^2 V}{\partial S^2} + b(t,S) \frac{\partial V}{\partial S} + c(t,S) V + f(t,S) = 0}
\end{equation*}
con una condición final
\begin{equation*}
    \boxed{V(T,S) = \phi(S)}
\end{equation*}
tal que
\begin{equation*}
    \boxed{a(t,0) = 0, \quad b(t,0) = 0}
\end{equation*}
y que en $S\to\infty$ se comporta de manera lineal:
\begin{equation*}
    \boxed{V(t, S) = A(t) S + B(t), \quad S \to \infty}
\end{equation*}
entonces para resolverlo numéricamente en los puntos $V(t_j, S_i) = V_j^i$ se debe en primer lugar calcular las constantes:
\begin{equation*}
    \boxed{\alpha = \frac{1}{\Delta S^2}, \qquad \beta = \frac{1}{\Delta S}, \qquad \gamma = \frac{1}{\Delta t}}
\end{equation*}
y los valores
\begin{equation*}
    \boxed{\eta_i = \alpha a_j^i, \qquad \varphi_i = 2\alpha a_j^i  + \beta b_j^i - c_j^i, \qquad \psi_i = \alpha a_j^i + \beta b_j^i}
\end{equation*}

A continuación calcular los valores en $S=0$ (las integrales se pueden calcular numéricamente):
\begin{equation*}
    \boxed{V_j^0 = V(t_j, 0) = \phi(0) e^{\int_{t_j}^T c(u, 0)\,du} + \int_{t_j}^T f(u, 0) e^{\int_u^T c(v, 0)\,dv}\,du}
\end{equation*}
y para cada instante de tiempo $j=N, N-1, \ldots, 0$ se deben resolver el siguiente sistema de ecuaciones:
\begin{equation*}
    \boxed{A V_j = B V_{j+1} - f_j + C}
\end{equation*}
donde las matrices $A$ y $B$ son:
\begin{align*}
    &\boxed{
        A = \begin{pmatrix}
            A_{1,1} & A_{1,2} & 0 & \cdots & 0 & 0 & 0 \\
            A_{2,1} & A_{2,2} & A_{2,3} & \cdots & 0 & 0 & 0 \\
            0 & A_{3,2} & A_{3,3} & \cdots & 0 & 0 & 0 \\
            \vdots & \vdots & \vdots & \ddots & \vdots & \vdots & \vdots \\
            0 & 0 & 0 & \cdots & A_{M-2,M-2} & A_{M-2,M-1} & 0 \\
            0 & 0 & 0 & \cdots & A_{M-1,M-2} & A_{M-1,M-1} & A_{M-1,M} \\
            0 & 0 & 0 & \cdots & 0  & \omega_{M,M-1} & \omega_{M,M}
        \end{pmatrix},
        \quad \begin{cases}
            A_{i,i-1} = \theta\eta_i \\
            A_{i,i} = -\gamma - \theta \varphi_i \\
            A_{i,i+1} = \theta\psi_i \\
            \omega_{M,M-1} = \theta(\eta_M - \psi_M) \\
            \omega_{M,M} = -\gamma - \theta (\varphi_M - 2\psi_M)
        \end{cases}
    } \\
    & \boxed{
        B = \begin{pmatrix}
            B_{1,1} & B_{1,2} & 0 & \cdots & 0 & 0 & 0 \\
            B_{2,1} & B_{2,2} & B_{2,3} & \cdots & 0 & 0 & 0 \\
            0 & B_{3,2} & B_{3,3} & \cdots & 0 & 0 & 0 \\
            \vdots & \vdots & \vdots & \ddots & \vdots & \vdots & \vdots \\
            0 & 0 & 0 & \cdots & B_{M-2,M-2} & B_{M-2,M-1} & 0 \\
            0 & 0 & 0 & \cdots & B_{M-1,M-2} & B_{M-1,M-1} & B_{M-1,M} \\
            0 & 0 & 0 & \cdots & 0  & \varpi_{M,M-1} & \varpi_{M,M}
        \end{pmatrix}
        \quad \begin{cases}
            B_{i,i-1} = -(1-\theta)\eta_i \\
            B_{i,i} = -\gamma + (1-\theta) \varphi_i \\
            B_{i,i+1} = -(1-\theta)\psi_i \\
            \varpi_{M,M-1} = -(1-\theta)(\eta_M - \psi_M) \\
            \varpi_{M,M} = -\gamma + (1-\theta) (\varphi_M - 2\psi_M)
        \end{cases}
    }
\end{align*}
y donde
\begin{align*}
    &\boxed{
        V_j = \begin{pmatrix}
            V_j^1 \\
            V_j^2 \\
            \vdots \\
            V_j^M
        \end{pmatrix}, \quad
        V_{j+1} = \begin{pmatrix}
            V_{j+1}^1 \\
            V_{j+1}^2 \\
            \vdots \\
            V_{j+1}^M
        \end{pmatrix}, \quad
        f_j = \begin{pmatrix}
            f_j^1 \\
            f_j^2 \\
            \vdots \\
            f_j^M
        \end{pmatrix}
    } \\
    &\boxed{
        C = \begin{pmatrix}
            -\theta\eta_1 V_j^{0} - (1-\theta)\eta_1 V_{j+1}^{0} \\
            0 \\
            \vdots \\
            0 \\
        \end{pmatrix}
        = \begin{pmatrix}
            -A_{1,0}V_j^{0} + B_{1,0}V_{j+1}^{0} \\
            0 \\
            \vdots \\
            0 \\
        \end{pmatrix}
    }
\end{align*}

Finalmente, se debe calcular el valor en $V_j^{M+1}$ usando la fórmula:
\begin{equation*}
    \boxed{V_j^{M+1} = 2V_j^{M} - V_j^{M-1}}
\end{equation*}











