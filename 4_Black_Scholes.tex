
\section{Modelos Black-Scholes}




\subsection{Modelo opciones básico}
Se construye cartera
\[
\Pi = V(S,t) - \Delta S
\]
sabiendo que
\[
dS = \mu Sdt + \sigma S d\mathnormal{X}
\]
entonces, por el lema de Itô~\ref{Ito} se tiene que
\[
d\Pi = \frac{\partial V}{\partial t}dt + \frac{\partial V}{\partial S}dS + \frac{\sigma^2S^2}{2} \frac{\partial^2 V}{\partial S^2}dt - \Delta dS
\]
que, haciendo un \textbf{delta hedging} $\Delta = \frac{\partial V}{\partial S}$ se obtiene que, sin arbitraje:
\[
d\Pi = \left( \frac{\partial V}{\partial t} + \frac{\sigma^2S^2}{2} \frac{\partial^2 V}{\partial S^2} \right)dt
\]
que debe ser igual que
\[
d\Pi = r\Pi dt = r\left( V - S \frac{\partial V}{\partial S} \right)dt
\]
por lo que
\[
\boxed{\frac{\partial V}{\partial t} + \frac{\sigma^2S^2}{2} \frac{\partial^2 V}{\partial S^2} + rS \frac{\partial V}{\partial S} -rV = 0}
\]

Se debe de tener en cuenta que esto calcula el valor justo de la opción, que es el valor actualizado de su payoff bajo una \textbf{risk-neutral random walk} para el subyacente. Este camino aleatorio es
\[
    dS = rSdt + \sigma S d\mathnormal{X}
\]
Por lo tanto no es lo mismo que la probabilidad de que la opción quede ITM calculada con lo expuesto en el apartado~\ref{sec:ExpectedFirstExitTimes}.


\subsection{Opciones de activos con dividendos continuos}
Tener comprado da un dividendo continuo de $DSdt$, luego la variación de la cartera es
\[
d\Pi = \frac{\partial V}{\partial t}dt + \frac{\partial V}{\partial S}dS + \frac{\sigma^2S^2}{2} \frac{\partial^2 V}{\partial S^2}dt - \Delta dS - D\Delta dS
\]
que usando igualmente que $\Delta = \frac{\partial V}{\partial S}$, se tiene que
\[
\boxed{\frac{\partial V}{\partial t} + \frac{\sigma^2S^2}{2} \frac{\partial^2 V}{\partial S^2} + (r-D)S \frac{\partial V}{\partial S} -rV = 0}
\]



\subsection{Currency options}
En vez de acciones como subyacente, se usa una moneda extranjera con un interés $r_f$ que se comporta como un dividendo continuo, por lo que queda
\[
\boxed{\frac{\partial V}{\partial t} + \frac{\sigma^2S^2}{2} \frac{\partial^2 V}{\partial S^2} + (r-r_f)S \frac{\partial V}{\partial S} -rV = 0}
\]




\subsection{Commodity options}
EL subyacente es un commodity, que tiene un coste de almacenamiento. En este caso se asume continuo $q$ (i.e. $qSdt$). Como es un coste se puede ver como un dividendo negativo, por lo que
\[
\boxed{\frac{\partial V}{\partial t} + \frac{\sigma^2S^2}{2} \frac{\partial^2 V}{\partial S^2} + (r+q)S \frac{\partial V}{\partial S} -rV = 0}
\]




\subsection{Forwards contracts}
Construyendo una cartera igual que BS clásico, se llega a la misma EDP.%
\[
\boxed{\frac{\partial V}{\partial t} + \frac{\sigma^2S^2}{2} \frac{\partial^2 V}{\partial S^2} + rS \frac{\partial V}{\partial S} -rV = 0}
\]
pero se debe añadir como condición final
\[\boxed{V(S,T)=S-\bar{S}}\]
donde $\bar{S}$ es el precio fijado (delivery price). Su solución es
\[
\boxed{S-\bar{S}e^{-r(T-t_0)}}
\]
Su delivery price es el que da valor 0 al contrato en un primer momento, luego 
\[
\bar{S} = S_0e^{r(T-t_0)}
\]
Su forward price, por otro lado, es  
\[
\text{Forward price} = Se^{r(T-t_0)}
\]




\subsection{Future contracts}
Como el valor del contrato se resetea a 0 todos los días (hay compensación diaria), el valor del contrato durante su vida es 0. Denotando por $F(S,t)$ al valor del contrato:
\begin{align*}
    \Pi &= F(S,T) - \Delta S = - \Delta S \\
    d\Pi &= dF(S,T) - d\Delta S \\
    &= \frac{\partial F}{\partial t}dt + \frac{\partial F}{\partial S}dS + \frac{\sigma^2S^2}{2} \frac{\partial^2 F}{\partial S^2}dt - \Delta dS
\end{align*}
luego tomando $\Delta = \frac{\partial F}{\partial S}$ y $d\Pi=r\Pi dt$ se obtiene
\[
\boxed{\frac{\partial F}{\partial t}dt + \frac{\sigma^2S^2}{2} \frac{\partial^2 F}{\partial S^2}dt + rS\frac{\partial F}{\partial S} = 0}
\]
con la condición final de que
\[
\boxed{F(S,T) = S}
\]
Su solución es
\[\boxed{F(S,t) = Se^{r(T-t)}}\]
por lo que para el caso de interés constante, los futures y los forwards valen lo mismo.






\subsection{Opciones sobre futuros}
Son opciones en las que el subyacente es un contrato de futuro. Se sabe que
\[
F = Se^{r(T-t)}
\]
por lo que, haciendo un cambio de variable $V(S,t) = W(F,t)$ se obtiene la EDP
\[
\boxed{\frac{\partial W}{\partial t}dt + \frac{\sigma^2}{2}F^2 \frac{\partial^2 W}{\partial F^2}dt - rW    = 0}
\]




\subsection{Condiciones de frontera y finales}
En una opción europea, se tienen las siguientes condiciones:
\begin{itemize}
    \item Condiciones temporales:
    \begin{itemize}
        \item Call:
        \[\boxed{C(S,T) = \max(S-E, 0)}\]
        \item Put:
        \[\boxed{C(S,T) = \max(E-S, 0)}\]
    \end{itemize}
    \item Condiciones de frontera: (se justificarán más adelante)
    \begin{itemize}
        \item Call:
        \[\boxed{C(0,t) = 0}\]
        \[\boxed{C(S,t) \xrightarrow{S\rightarrow\infty}  S - Ee^{-r(T-t)}}\]
        \item Put:
        \[\boxed{P(0,t) = Ee^{-r(T-t)}}\]
        \[\boxed{P(S,t) \xrightarrow{S\rightarrow\infty}  0}\]
    \end{itemize}
\end{itemize}






\subsection{Algunas propiedades de las opciones europeas}

\begin{remark}\label{ActPayoff}
    Si el payoff de una cartera es mayor o igual a $M$, entonces, en ausencia de arbitraje, el valor actual de la cartera es mayor o igual que el valor actualizado:
    \[\boxed{\Pi(T) \geq M \Rightarrow \Pi(t) \geq Me^{-r(T-t)}}\]
    Si no fuese el caso, se podría pedir al banco una cantidad $Me^{-r(T-t)}$ en tiempo $t$ y comprar la cartera. Entonces, en tiempo $T$ se pagaría el préstamo con el payoff y se generaría beneficio.
\end{remark}


\begin{proposition}\label{PropsCall}
    Sea $C(S,t)$ una opción Call europea con fecha de ejercicio $T$ y subyacente $S$ sin dividendos; entonces, en ausencia de arbitraje:
    \begin{enumerate}
        \item $\boxed{C \leq S}$
        \item $\boxed{C \geq \max(S-Ee^{-r(T-t)}, 0)}$
        \item $\boxed{0 \leq C_1 -C_2 \leq (E_2-E_1)e^{-r(T-t)}$ con $E_1<E_2}$.
    \end{enumerate}
\end{proposition}
\begin{proof}
    Utilizando la observación~\ref{ActPayoff}:
    \begin{enumerate}
        \item Sea la cartea $\Pi = S-C$, entonces
        \begin{align*}
            \Pi(T) &= S - \max(S - E,0) \\
            &= \left\{
            \begin{array}{ll}
              S,       & 0 \leq S \leq E \\
              E,        & S \geq E
            \end{array}
            \right\} \geq 0 \xRightarrow{\ref{ActPayoff}} \\
            \xRightarrow{\ref{ActPayoff}} \Pi(t) &= S-C \geq 0 \Rightarrow \\
            \Rightarrow S &\geq 0
        \end{align*}
        \item Sea la cartea $\Pi = S-C$, entonces
        \begin{align*}
            \Pi(T) &= S - \max(S - E,0) \\
            &= \left\{
            \begin{array}{ll}
              S,       & 0 \leq S \leq E \\
              E,        & S \geq E
            \end{array}
            \right\} \leq E \xRightarrow{\ref{ActPayoff}}\\
            \xRightarrow{\ref{ActPayoff}} \Pi(t) &= S - C \leq Ee^{-r(T-t)} \\
            \Rightarrow C &\geq  S - Ee^{-r(T-t)} \xRightarrow{C \geq 0} \\
            \xRightarrow{C \geq 0} C &\geq \max(S - Ee^{-r(T-t)}, 0)
        \end{align*}
        \item Sea la cartea $\Pi = C_1 - C_2$, entonces
        \begin{align*}
            \Pi(T) &= \max(S - E_1,0) - \max(S - E_2,0) \\
            &= \left\{
            \begin{array}{ll}
            0,       & 0 \leq S < E_1 \\
            S-E_1,   & E_1 \leq S < E_2 \\
            E_2-E_1, & S \geq E_2
            \end{array}
            \right\} \Rightarrow \\
            0 &\leq \Pi(T) \leq E_2-E_1 \xRightarrow{\ref{ActPayoff}}\\
            \xRightarrow{\ref{ActPayoff}} 0 &\leq \Pi(t) \leq (E_2-E_1)e^{-r(T-t)} \Rightarrow \\
            \Rightarrow 0 &\leq C_1 - C_2 \leq (E_2-E_1)e^{-r(T-t)}
        \end{align*}
    \end{enumerate}
\end{proof}




\begin{proposition}
    Sea $P(S,t)$ una opción Put europea con fecha de ejercicio $T$ y subyacente $S$ sin dividendos; entonces, en ausencia de arbitraje:
    \begin{enumerate}
        \item $\boxed{P \leq Ee^{-r(T-t)}}$
        \item $\boxed{P \geq Ee^{-r(T-t)} - S}$
        \item $\boxed{0 \leq P_2 - P_1 \leq (E_2-E_1)e^{-r(T-t)}$ con $E_1<E_2}$.
    \end{enumerate}
\end{proposition}
\begin{proof}
    Utilizando la observación~\ref{ActPayoff}:
    \begin{enumerate}
        \item Sea la cartea $\Pi = P - E$, entonces
        \begin{align*}
            \Pi(T) &= \max(E - S,0) - E \\
            &= \left\{
            \begin{array}{ll}
              -S,       & 0 \leq S < E \\
              -E,        & S \geq E
            \end{array}
            \right\} \leq 0 \xRightarrow{\ref{ActPayoff}}\\
            \xRightarrow{\ref{ActPayoff}} \Pi(t) &= P - Ee^{-r(T-t)} \leq 0 \\
            \Rightarrow P &\leq  Ee^{-r(T-t)}
        \end{align*}
        \item Sea la cartea $\Pi = S + P$, entonces
        \begin{align*}
            \Pi(T) &= S +\max(E - S,0) \\
            &= \left\{
            \begin{array}{ll}
              E,       & 0 \leq S < E \\
              S,        & S \geq E
            \end{array}
            \right\} \geq E \xRightarrow{\ref{ActPayoff}}\\
            \xRightarrow{\ref{ActPayoff}} \Pi(t) &= S + P \geq Ee^{-r(T-t)} \\
            \Rightarrow P &\geq  Ee^{-r(T-t)} - S
        \end{align*}
        \item Sea la cartea $\Pi = P_2 - P_1$, entonces
        \begin{align*}
            \Pi(T) &= \max(E_2 - S,0) - \max(E_1 - S,0) \\
            &= \left\{
            \begin{array}{ll}
              E_2 - E_1,       & 0 \leq S < E_1 \\
              E_2-S,        & E_1 \leq S < E_2 \\
              0,        & S \geq E_2
            \end{array}
            \right\} \Rightarrow \\
            0 &\leq \Pi(T) \leq E_2-E_1 \xRightarrow{\ref{ActPayoff}}\\
            \xRightarrow{\ref{ActPayoff}} 0 &\leq \Pi(t) \leq (E_2-E_1)e^{-r(T-t)} \Rightarrow \\
            \Rightarrow 0 &\leq P_2-P_1 \leq (E_2-E_1)e^{-r(T-t)}
        \end{align*}
    \end{enumerate}
\end{proof}





\begin{proposition}
    Sea $P(S,t)$ una opción Put europea con fecha de ejercicio $T$ y subyacente $S$ sin dividendos; entonces, en ausencia de arbitraje:
    \begin{enumerate}
        \item $\boxed{C_A \geq C_B}$ donde $C_a, C_B$ son Calls europeas con precio de ejercicio $E$ y fechas de ejercicio $T_A, T_B$ tal que $T_A > T_B$.
        \item $\boxed{C_2 \leq \frac{E_3-E_2}{E_3-E_1}C_1 + \frac{E_2-E_1}{E_3-E_1}C_3}$ donde $C_1, C_2, C_3$ son Calls europeas con fecha de ejercicio $T$ y strikes $E_1, E_2, E_3$ donde $E_1<E_2<E_3$.
    \end{enumerate}
\end{proposition}
\begin{proof}
    Utilizando la observación~\ref{ActPayoff}:
    \begin{enumerate}
        \item Sea la cartera $\Pi = C_A-C_B$, entonces
        \begin{align*}
            \Pi(T_B) &= C_A(S,T_B) - \max(S-E,0) \overset{\ref{PropsCall}}{\geq} \\
            &\overset{\ref{PropsCall}}{\geq} \max(S-Ee^{-r(T-t)}) - \max(S-E,0) \geq \\
            &\geq \max(S-E) - \max(S-E,0) = 0 \\
            &\Rightarrow \Pi(T_B) \geq 0 \Rightarrow \\
            &\Rightarrow \Pi(t) = C_A-C_B \geq 0 \Rightarrow \\
            &\Rightarrow C_A \geq C_B
        \end{align*}
        \item Sea la cartera $\Pi = - C_2 + \lambda C_1 + (1-\lambda) C_3 $ y se considera 
        \begin{align*}
            &E_2 = \lambda E_1 + (1-\lambda) E_3 = \lambda E_1 + E_3-\lambda E_3 = E_3 + \lambda(E_1-E_3) \Rightarrow \\
            \Rightarrow &\lambda = \frac{E_2-E_3}{E_1-E_3} = \frac{E_3-E_2}{E_3-E_1} \Rightarrow \\
            \Rightarrow &(1-\lambda) = \frac{E_2-E_1}{E_3-E_1}
        \end{align*}
        entonces
        \begin{align*}
            \Pi(T) &= -\max(S-E_2, 0) + \lambda\max(S-E_1,0) + (1+\lambda)\max(S-E_3, 0) = \\
            &=\left\{
            \begin{array}{ll}
              0,       & 0 \leq S < E_1 \\
              \lambda(S-E_1),        & E_1 \leq S < E_2 \\
              - (S-E_2) +\lambda(S-E_1),        & E_2 \leq S < E_3 \\
              - (S-E_2) +\lambda(S-E_1) + (1-\lambda)(S-E_3),        & S \geq E_3
            \end{array}
            \right\} \\
            &=\left\{
            \begin{array}{ll}
              0,       & 0 \leq S < E_1 \\
              \frac{E_3-E_2}{E_3-E_1}(S-E_1),        & E_1 \leq S < E_2 \\
              (1-\lambda)(E_3+S),        & E_2 \leq S < E_3 \\
              (1-\lambda)(E_3+S) + (1-\lambda)(S-E_3),        & S \geq E_3
            \end{array}
            \right\} \\
            &=\left\{
            \begin{array}{ll}
              0,       & 0 \leq S < E_1 \\
              \lambda(S-E_1),        & E_1 \leq S < E_2 \\
              (1-\lambda)(E_3+S),        & E_2 \leq S < E_3 \\
              (1-\lambda)2S,        & S \geq E_3
            \end{array}
            \right\} \\
            &=\left\{
            \begin{array}{ll}
              0,       & 0 \leq S < E_1 \\
              \frac{E_3-E_2}{E_3-E_1}(S-E_1),        & E_1 \leq S < E_2 \\
              \frac{E_2-E_1}{E_3-E_1}(E_3+S),        & E_2 \leq S < E_3 \\
              \frac{E_2-E_1}{E_3-E_1}2S,        & S \geq E_3
            \end{array}
            \right\} \geq 0 \xRightarrow{\ref{ActPayoff}} \\
            \xRightarrow{\ref{ActPayoff}} \Pi(t) &= - C_2 + \lambda C_1 + (1-\lambda) C_3 \geq 0 \\
            C_2 &\geq \frac{E_3-E_2}{E_3-E_1} C_1 + \frac{E_2-E_1}{E_3-E_1} C_3
        \end{align*}
    \end{enumerate}
\end{proof}







