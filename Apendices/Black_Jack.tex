\section{Blackjack y apuestas}\label{sec:blackjack}


\subsection{Reglas del blackjack}
El objetivo es sumar 21 sin pasarse; los ases valen 1 o 11 (según quiera el jugador), las figuras valen 10 y el resto su valor nominal. 

Antes de repartir se debe apostar algo. A continuación, se reparten dos cartas por persona (boca abajo) y el dealer se reparte dos cartas: una boca arriba y otra boca abajo. Un \textbf{blackjack} o \textbf{natural} es sacar un as y una figura con las dos primeras cartas y se gana automáticamente, a menos que el dealer también saque un natural, en cuyo caso se empata. Un natural ganador paga al jugador 3 a 2. 

Una vez repartidas las cartas se reparte en sentido de las agujas del reloj, comenzando por el jugador a la izquierda del dealer. El jugador puede pedir carta (hit) o plantarse (stand). Si se pasa de 21, pierde automáticamente. 
El jugador también puede hacer otras acciones:
\begin{itemize}
    \item \textbf{Double down}: consiste en duplicar la apuesta hecha antes de repartir las cartas y recibir una sola carta más.
    \item \textbf{Splitting pairs}: si las dos primeras cartas son del mismo valor, se puede dividir y jugar con dos manos, apostando el mismo valor en cada una. Los ases solo puede recibir una carta adicional. Tras dividir, el as más figura cuentan como 21, no como blackjack. 
    \item Si la carta descubierta del dealer es un as, el jugador puede hacer \textbf{insurance} (seguro), apostando la mitad de la apuesta original. Si la carta oculta del dealer es un 10, el jugador gana 2 a 1.
\end{itemize}
En ocasiones, se permite \textbf{rendir} (surrender), lo que significa que el jugador puede abandonar la mano y perder solo la mitad de su apuesta. 

El dealer no tiene decisiones que tomar: debe pedir carta hasta que tenga 16 y plantarse a partir de 17. 

En un empate el dinero no cambia de sitio y se guarda para la siguiente partida.



Existen estrategias concretas para ganar en el blackjack, pero se resume en lo siguiente:
\begin{itemize}
    \item Si vas sin estrategia pierdes el dinero rápidamente.
    \item Si copias al dealer, la casa tiene un 5\% o 6\% de ventaja.
    \item La mejor estrategia consiste en saber cuando hacer que acción, que te dará una ligera ventaja sobre la casa.
\end{itemize}





