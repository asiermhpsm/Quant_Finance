\section{Value at Risk (VaR)}
Value at Risk es una estimación, con un cierto grado de confianza, de cuánto se puede perder de la cartera en un horizonte temporal determinado.

Por ejemplo decir que en la siguiente semana hay una 95\% de probabilidad de que no se perderán más de 10 millones de euros se escribe como:
\[
    \mathbb{P} \left( \delta V \leq -10M \right) = 0.05
\]
donde $\delta V$ es la variación del valor de la cartera. Con símbolos:
\[
    \mathbb{P} \left( \delta V \leq \text{VaR} \right) = 1 - c
\]
donde $c$ es el grado de confianza. El value at risk no tiene en cuenta condiciones extremas, por lo que es válido solo para operaciones del día a día. Para el cálculo del VaR se requiere los precios actuales de los activos de la cartera y sus volatilidades y correlaciones. Normalmente se asume que los movimientos de la cartera son aleatorios y siguen una distribución normal.



\subsection{VaR de un solo activo}
Se tiene una cantidad $\Delta$ de un activo con precio $S$ y volatilidad $\sigma$. Se quiere saber con una confianza $c$ cual es la pérdida máxima esperada en un horizonte temporal $\delta t$. Se asume que el precio del activo sigue un proceso estoástico basado en la distribución normal:
\[
    dS = \mu S dt + \sigma S dW
\]

Suponiendo que $\delta t$ es pequeño, se puede ignorar la media por lo que
\[
    dS \approx \sigma S \sqrt{\delta t} Z
\]
siendo $Z \sim \mathcal{N}(0,1)$, por lo que
\[
    \text{std}(dS) = \sigma S \sqrt{\delta t}
\]
y como la cartera tiene una cantidad $\Delta$ del activo, la desviación estándar de la variación del valor de la cartera es
\[
    \sigma \Delta S \sqrt{\delta t}
\]

Se quiere calcular la posición de la cola izquierda de la distribución correspondiente a $1-c$. Por lo tanto, el VaR de un activo es
\[
    \boxed{\text{VaR} = -\sigma \Delta S \sqrt{\delta t} \cdot \alpha(1-c)}
\]
siendo $\alpha(\cdot)$ la función inversa de distribución acumulativa de la normal estándar. En la tabla \ref{tab:inv_norm_table} se muestran los valores de la función inversa de la distribución normal estándar para diferentes niveles de confianza y $c$.
\begin{table}[H]
    \centering
    \small
    \begin{tabular}{|c|c|c|c|c|c|c|c|c|c|c|}
        \hline
        \textbf{Confi} & \textbf{0.000} & \textbf{0.001} & \textbf{0.002} & \textbf{0.003} & \textbf{0.004} & \textbf{0.005} & \textbf{0.006} & \textbf{0.007} & \textbf{0.008} & \textbf{0.009} \\
        \hline
        \textbf{0.80} & 0.84162 & 0.84520 & 0.84879 & 0.85239 & 0.85600 & 0.85962 & 0.86325 & 0.86689 & 0.87055 & 0.87422 \\
        \hline
        \textbf{0.81} & 0.87790 & 0.88159 & 0.88529 & 0.88901 & 0.89273 & 0.89647 & 0.90023 & 0.90399 & 0.90777 & 0.91156 \\
        \hline
        \textbf{0.82} & 0.91537 & 0.91918 & 0.92301 & 0.92686 & 0.93072 & 0.93459 & 0.93848 & 0.94238 & 0.94629 & 0.95022 \\
        \hline
        \textbf{0.83} & 0.95417 & 0.95812 & 0.96210 & 0.96609 & 0.97009 & 0.97411 & 0.97815 & 0.98220 & 0.98627 & 0.99036 \\
        \hline
        \textbf{0.84} & 0.99446 & 0.99858 & 1.00271 & 1.00686 & 1.01103 & 1.01522 & 1.01943 & 1.02365 & 1.02789 & 1.03215 \\
        \hline
        \textbf{0.85} & 1.03643 & 1.04073 & 1.04505 & 1.04939 & 1.05374 & 1.05812 & 1.06252 & 1.06694 & 1.07138 & 1.07584 \\
        \hline
        \textbf{0.86} & 1.08032 & 1.08482 & 1.08935 & 1.09390 & 1.09847 & 1.10306 & 1.10768 & 1.11232 & 1.11699 & 1.12168 \\
        \hline
        \textbf{0.87} & 1.12639 & 1.13113 & 1.13590 & 1.14069 & 1.14551 & 1.15035 & 1.15522 & 1.16012 & 1.16505 & 1.17000 \\
        \hline
        \textbf{0.88} & 1.17499 & 1.18000 & 1.18504 & 1.19012 & 1.19522 & 1.20036 & 1.20553 & 1.21073 & 1.21596 & 1.22123 \\
        \hline
        \textbf{0.89} & 1.22653 & 1.23186 & 1.23723 & 1.24264 & 1.24808 & 1.25357 & 1.25908 & 1.26464 & 1.27024 & 1.27587 \\
        \hline
        \textbf{0.90} & 1.28155 & 1.28727 & 1.29303 & 1.29884 & 1.30469 & 1.31058 & 1.31652 & 1.32251 & 1.32854 & 1.33462 \\
        \hline
        \textbf{0.91} & 1.34076 & 1.34694 & 1.35317 & 1.35946 & 1.36581 & 1.37220 & 1.37866 & 1.38517 & 1.39174 & 1.39838 \\
        \hline
        \textbf{0.92} & 1.40507 & 1.41183 & 1.41865 & 1.42554 & 1.43250 & 1.43953 & 1.44663 & 1.45381 & 1.46106 & 1.46838 \\
        \hline
        \textbf{0.93} & 1.47579 & 1.48328 & 1.49085 & 1.49851 & 1.50626 & 1.51410 & 1.52204 & 1.53007 & 1.53820 & 1.54643 \\
        \hline
        \textbf{0.94} & 1.55477 & 1.56322 & 1.57179 & 1.58047 & 1.58927 & 1.59819 & 1.60725 & 1.61644 & 1.62576 & 1.63523 \\
        \hline
        \textbf{0.95} & 1.64485 & 1.65463 & 1.66456 & 1.67466 & 1.68494 & 1.69540 & 1.70604 & 1.71689 & 1.72793 & 1.73920 \\
        \hline
        \textbf{0.96} & 1.75069 & 1.76241 & 1.77438 & 1.78661 & 1.79912 & 1.81191 & 1.82501 & 1.83842 & 1.85218 & 1.86630 \\
        \hline
        \textbf{0.97} & 1.88079 & 1.89570 & 1.91104 & 1.92684 & 1.94313 & 1.95996 & 1.97737 & 1.99539 & 2.01409 & 2.03352 \\
        \hline
        \textbf{0.98} & 2.05375 & 2.07485 & 2.09693 & 2.12007 & 2.14441 & 2.17009 & 2.19729 & 2.22621 & 2.25713 & 2.29037 \\
        \hline
        \textbf{0.99} & 2.32635 & 2.36562 & 2.40892 & 2.45726 & 2.51214 & 2.57583 & 2.65207 & 2.74778 & 2.87816 & 3.09023 \\
        \hline
    \end{tabular}
    \caption{Tabla de valores de la función inversa de la distribución normal estándar}
    \label{tab:inv_norm_table}
\end{table}



El desprecio de la media solo es válido para horizontes temporales pequeños. Si el horizonte temporal es grande, la media no se puede despreciar y el VaR se calcula como:
\[
    \boxed{\text{VaR} = \Delta S \left( \mu \delta t - \sigma \sqrt{\delta t} \cdot \alpha(1-c) \right)}
\]





\subsection{VaR de una cartera de activos}
Para el caso de una cartera, se puede probar que el VaR es
\[
    \boxed{\text{VaR} = -\alpha (1-c) \, \delta t^{1/2} \sqrt{ \sum_{i=1}^{M} \sum_{j=1}^{M} \Delta_i \Delta_j \sigma_i \sigma_j \rho_{ij} S_i S_j }}
\]





\subsection{Carteras con derivados}
Para este tipo de carteras, aunque el subyacente siga tenga un comportamiento de una distribución normal, el valor de los derivados no tiene porque seguir una distribución normal, por lo que puede ser complicado. Existen distintas estrategias.


\subsubsection{La aproximación delta}
Útil para marcos temporales pequeños.

La delta $\Delta$ mide la sensisbilidad de la opción al precio del subyacente. Si la desviación estándar del subyacente es $\sigma S \sqrt{\delta t}$, entonces la desviación estandar de la opción es
\[
    \sigma S \sqrt{\delta t} \Delta
\]

EN el caso de una cartera de opciones la $\Delta$ debe ser de la pocisión entera, i.e.\ la suma de las deltas de todas las opciones sobre el mismo subyacente. Si el marco temporal es pequeño, una aproximación del VaR de la cartera es
\[
    \boxed{\text{VaR} = -\alpha (1-c) \, \delta t^{1/2} \sqrt{ \sum_{i=1}^{M} \sum_{j=1}^{M} \Delta_i \Delta_j \sigma_i \sigma_j \rho_{ij} S_i S_j }}
\]
donde $\Delta_i$ es el ratio de cambio de la cartera con respecto a i-ésimo activo.

Normalmente, la volatilidad que se usa es la implícita.




\subsubsection{La aproximación delta-gamma}
Se puede usar para marcos temporales más grandes. 

Suponiendo que la cartera consiste en una opción con una acción como subyacente, entonces la relación entre el cambio del subyacente $\delta S$ y el cambio de la opción $\delta V$, mediante una expansión de Taylor, es
\[
\delta V = \frac{\partial V}{\partial S} \delta S + \frac{1}{2} \frac{\partial^2 V}{\partial S^2} (\delta S)^2 + \frac{\partial V}{\partial t} \delta t + \cdots
\]
Asumiendo que
\[
\delta S = \mu S \delta t + \sigma S \delta t^{1/2} Z
\]
donde $Z \sim \mathcal{N}(0,1)$, se puede escribir ignorando los términos de orden mayor como
\begin{align*}
    \delta V &= \frac{\partial V}{\partial S} \sigma S \, \delta t^{1/2} Z + \delta t \left( \frac{\partial V}{\partial S} \mu S + \frac{1}{2} \frac{\partial^2 V}{\partial S^2} \sigma^2 S^2 Z^2 + \frac{\partial V}{\partial t} \right) + \cdots \Rightarrow \\
     &= \boxed{\Delta \sigma S \, \delta t^{1/2} Z + \delta t \left( \Delta \mu S + \frac{1}{2} \Gamma \sigma^2 S^2 Z^2 + \Theta \right)} + \cdots
\end{align*}

Si esto se ve como una función $f(Z)$, entonces $\delta V$ cumple con las siguientes restricciones:
\[
    \left\{
    \begin{aligned}
        \delta V \geq -\frac{\Delta^2}{2\Gamma} &\Rightarrow \min(\delta V) = -\frac{\Delta^2}{2\Gamma}, \qquad \Gamma > 0 \\
        \delta V \leq -\frac{\Delta^2}{2\Gamma} &\Rightarrow \max(\delta V) = -\frac{\Delta^2}{2\Gamma}, \qquad \Gamma < 0.
    \end{aligned}
    \right\}
\]
que se da cuando
\[
    Z = -\frac{\Delta}{\Gamma \sigma S \delta t^{1/2}}
\]

Como ya no se está trabajando con una distribución normal, si no una distribución cuadrática de una normal, ya no se puede usar la función inversa de la normal estándar. En su lugar, se suelen usar simulaciones.












