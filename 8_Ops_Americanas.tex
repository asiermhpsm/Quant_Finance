
\section{Opciones americanas}

Aquellas que pueden ejercerse en cualquier momento antes de su vencimiento. Sea $\Phi(S)$ el payoff de la opción, entonces se cumple que:
\[
\boxed{V \geq \Phi(S)}
\]
Si no fuese el caso, entonces $V < \Phi(S) \Rightarrow \Phi(S) - V > 0$, por lo que se podría comprar una opción y ejercer al momento obteniendo un beneficio sin riesgo, por lo que habría arbitraje.

En general, el punto óptimo de ejercicio es aquel que hace que la pendiente de la opción sea la misma que la pendiente del \textit{payoff}.

Sea la cartera
\[
\Pi = V - \Delta S
\]
entonces la diferencia entre el cambio de valor de la cartera y su crecimiento por el interés es:
\[
d\Pi - r\Pi dt = \left( \frac{\partial V}{\partial t} + \frac{1}{2} \sigma^2 S^2 \frac{\partial^2 V}{\partial S^2} - r(V - \Delta S) \right) dt + \left( \frac{\partial V}{\partial S} - \Delta \right) dS
\]
y eligiendo $\Delta = \frac{\partial V}{\partial S}$, se obtiene:
\[
d\Pi - r\Pi dt = \left( \frac{\partial V}{\partial t} + \frac{1}{2} \sigma^2 S^2 \frac{\partial^2 V}{\partial S^2} - rV \right) dt
\]
Para evitar arbitraje solo se debe asegurar que esta resta sea no positiva, de manera que la cartera no gane mas que el interés:
\[
\boxed{\frac{\partial V}{\partial t} + \frac{1}{2} \sigma^2 S^2 \frac{\partial^2 V}{\partial S^2} - rV \leq 0}
\]
El caso $\cdots < 0$ no produce arbitraje pq al vender una opción y meterlo en el banco (para que crezca a ritmo $r$), se corre el riesgo de que el comprador ejerza en cualquier momento.







