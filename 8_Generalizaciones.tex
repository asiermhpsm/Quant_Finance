
\section{Generalizaciones sencillas del modelo}

\subsection{Dividendos discretos}
Para evitar arbitraje, el precio de una acción con dividendo $D_i$ en el instante $t_i$ debe debe cumplir la condición de salto:
\[
    S(t_i^+) = S(t_i^-) - D_i
\]
luego la opción que tenga esa acción como suyacente debe cumplir que
\[
    V(S(t_i^-), t_i^-) = V(S(t_i^+), t_i^+) \Rightarrow \boxed{V(S, t_i^-) = V(S - D_i, t_i^+)}
\]\label{eq:divs_discretos}
El valor de la opción no pega un salto en el instante $t_i$ y es continua.




\subsection{Préstamo de acciones}
Cuando se habla de ir \textit{short} en una acción, muchas veces no se tiene por lo que se pide prestada. Pero este préstamo tiene un coste de interés $R$ sobre el valor de la acción. Usando el mismo argumento que para el modelo básico de Black-Scholes:
\begin{align*}
    &\left\{ 
        \begin{array}{rcl} 
            dS &= \mu Sdt + \sigma S d\mathnormal{X} &\\ 
            \Pi &= V(S,t) - \Delta S &\Rightarrow d\Pi = dV- \Delta dS
        \end{array} 
    \right\} \Rightarrow \\
    \Rightarrow &d\Pi = \frac{\partial V}{\partial t}dt + \frac{\partial V}{\partial S}dS + \frac{\sigma^2S^2}{2} \frac{\partial^2 V}{\partial S^2}dt - \Delta dS
\end{align*}
Pero como ahora se tiene que pagar un interés por el préstamo de la acción:
\[
    d\Pi = \frac{\partial V}{\partial t}dt + \frac{\partial V}{\partial S}dS + \frac{\sigma^2S^2}{2} \frac{\partial^2 V}{\partial S^2}dt - \Delta dS - R\max(\Delta, 0)Sdt
\]
que, haciendo un \textbf{delta hedging} $\Delta = \frac{\partial V}{\partial S}$ se obtiene que, sin arbitraje:
\[
d\Pi = \left( \frac{\partial V}{\partial t} + \frac{\sigma^2S^2}{2} \frac{\partial^2 V}{\partial S^2} -RS\max\left(\frac{\partial V}{\partial S}, 0\right) \right)dt
\]
igualando a que $d\Pi = r\Pi dt$ se obtiene que
\[
\boxed{\frac{\partial V}{\partial t} + \frac{\sigma^2S^2}{2} \frac{\partial^2 V}{\partial S^2} + rS \frac{\partial V}{\partial S} -rV -RS\max\left(\frac{\partial V}{\partial S}, 0\right) = 0}
\]


\subsection{Parámetros dependientes del tiempo}
Para el caso de una opción europea, resolver el problema con parametros $r(t), D(t), \sigma(t)$ es lo mismo que reolverlo con los parámetros constantes
\[
    \boxed{
        \begin{aligned}
            r_c &= \frac{1}{T-t} \int_t^T r(\tau) d\tau \\
            D_c &= \frac{1}{T-t} \int_t^T D(\tau) d\tau \\
            \sigma_c^2 &= \frac{1}{T-t} \int_t^T \sigma^2(\tau) d\tau
        \end{aligned}
    }
\]
Para el caso de opciones americanas o exóticas se deben estudiar las condiciones de frontera.







\subsection{Power options y log contracts}
Se puede encontrar más información en el apéndice~\ref{ApexGenerals}.









